% Consistency testing.
%%%%%%%%%%%%%%%%%%%%%%

\chapter{Consistency testing}
\index{consistency testing|textbf}


In spin relaxation, datasets are often recorded at different magnetic fields. This is especially important when $R_2$ values are to be used since $\mu$s-ms motions contribute to $R_2$. This contribution being scaled quadratically with the strength of the magnetic field, recording at multiple magnetic fields helps extract it. Also, acquiring data at multiple magnetic fields allows overdetermination of the mathematical problems, e.g. in the model-free approach.

Recording at multiple magnetic fields is a good practice. However, it can cause artifacts if those different datasets are inconsistent. Inconsistencies can originate from, inter alia, the sample or the acquisition. Sample variations can be linked to changes in temperature, concentration, pH, etc. Water suppression is the main cause of acquisition variations as it affect relaxation parameters (especially NOE) of exposed and exchangeable moieties (e.g. the NH moiety).

It is thus a good idea to assess consistency of datasets acquired at different magnetic fields. For this purpose, three tests are implemented in relax. They are all based on the same principle : calculate a field independant value and compare it from one field to another.

The three tests are :

$J(0)$ : The spectral density at the zero frequency calculated using the reduced spectral density approach.

$F_\eta$ : A consistency function proposed by Fushman D. et al. (1998) JACS, 120: 10947-10952.

$F_{R_2}$ : A consistency function proposed by Fushman D. et al. (1998) JACS, 120: 10947-10952.

Different methods exist to compare tests values calculated from one field to another. These include correlation plots and histograms, and calculation of correlation, skewness and kurtosis coefficients.



Until this chapter is completed written please look at the sample script \file{consistency\_tests.py}.
