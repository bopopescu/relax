% Development of relax chapter.
%%%%%%%%%%%%%%%%%%%%%%%%%%%%%%%

\chapter{Development of relax}

This chapter is written for developers or those who would like to extend the functionality of relax.  It is not required for using relax.  If you would like to modify relax to suit your needs, please subscribe to all the relax mailing lists (see the open source infrastructure chapter for more details).  Announcements are sent to ``relax-announce at gna.org'' while ``relax-users at gna.org'' is the list where discussions about the usage of relax should be posted.  ``relax-devel at gna.org'' is where all discussions about the development of relax, including feature requests, program design, or any other discussions relating to relax's structure or code should be posted.  Finally, ``relax-commits at gna.org'' is where all changes to relax's code and documentation, as well as changes to the web pages, are automatically sent to.  Anyone interested in joining the project should subscribe to all four lists.



% Version control using Subversion.
%~~~~~~~~~~~~~~~~~~~~~~~~~~~~~~~~~~

\section{Version control using Subversion}\label{svn repository}

The development of relax requires the use of the Subversion (SVN) version control software \href{http://subversion.tigris.org/}{http://subversion.tigris.org/}.  The source code to relax is stored in an SVN repository located at \href{http://svn.gna.org/svn/relax/}{http://svn.gna.org/svn/relax/}.  Every single change ever made to the program is recorded in this repository, for more information see the open source infrastructure chapter.

Although the downloadable distribution archives can be modified, it is best that the most current and up to date revision, the \textit{head} revision, is modified instead.  More information about the basics of version control and how this is implemented in Subversion can be found in the Subversion book located at \href{http://svnbook.red-bean.com/}{http://svnbook.red-bean.com/}.

If you are not currently a relax developer you can check out the head revision, assuming that 1.2 is the current major version number, by typing

\example{\$ svn co svn://svn.gna.org/svn/relax/1.2 relax}

Otherwise if you are a developer, type

\example{\$ svn co svn+ssh://user\_name@svn.gna.org/svn/relax/1.2 relax}

replacing \texttt{user\_name} with your Gna!\ login name.  If your version is out of date, it can be updated to the latest revision by typing

\example{\$ svn up}

Modifications can be made to these sources.



% Coding conventions.
%~~~~~~~~~~~~~~~~~~~~

\section{Coding conventions}

The following conventions must be followed at all times for any code to be accepted into the relax repository.



% Indentation.
\subsection{Indentation}

\index{indentation|textbf}
Indentation should be set to four spaces rather than a tab character.  This is the recommendation given in the python style guide found at \href{http://www.python.org/doc/essays/styleguide.html}{http://www.python.org/doc/essays/styleguide.html}.  Emacs should automatically set the tabstop correctly.  For vi, add the following lines to \texttt{`$\sim$/.vimrc'}:

\begin{exampleenv}
set tabstop=4 \\
set shiftwidth=4 \\
set expandtab
\end{exampleenv}

Certain versions of vim, those within the 6.2 series, contain a bug where the tabstop value cannot be changed using the \texttt{`$\sim$/.vimrc'} file (although typing \texttt{`:set tabstop=4'} in vim will fix it).  One solution is to edit the file \texttt{`python.vim'} which on GNU/Linux systems is located in the path \texttt{`/usr/share/vim/ftplugin/'}.  It contains the two lines

\begin{exampleenv}
" Python always uses a `tabstop' of 8. \\
setlocal ts=8
\end{exampleenv}

If these lines are deleted, the bug will be removed.  Another way to fix the problem is to install newer versions of the run-time files (which will do the same thing).



% Doc strings.
\subsection{Doc strings}
\index{doc string|textbf}

These should be set to no more than 100 characters long including all leading white space.  The standard Python convention of a one line description separated from a detailed description by an empty line should be adhered to.  All functions should have a docstring describing in detail the structure and organisation of the code.



% Variable, function, and class names.
\subsection{Variable, function, and class names}

In relax, a mixture of both camel case (eg. CamelCase) and lower case with underscores is used.  Despite the variability, there are fixed rules which should be adhered to.  These naming conventions should be observed at all times.


% Variables and functions.
\subsubsection{Variables and functions}

For both variables and functions, lower case with underscores between words is always used.  This is for readability as the convention is much more fluent than camel case.  A few rare exceptions exist, an example is the Brownian diffusion tensor parameter of anisotropy, $\Diff_a$, which is referenced as \texttt{self.relax.data.diff[run].Da}.  As a rule though, all new variable or function names should be kept as lower case.


% Classes.
\subsubsection{Classes}

For classes, relax uses a mix of camel case (for example all the \texttt{RelaxError} objects) and underscores (for example \texttt{Model\_free}).  The first letter in all cases is always capitalised.  Generally the camel case is reserved for very low level classes which are involved in the program's infrastructure.  Examples include the RelaxError code, the threading code, and the \texttt{self.relax.data} code.  All the data analysis specific code, generic code, interface code, etc. uses underscores between the words with only the first letter capitalised.  One exception is the space mapping class \texttt{OpenDX}, the reason being that the program is called `OpenDX'.



% Submitting changes to the relax project.
%~~~~~~~~~~~~~~~~~~~~~~~~~~~~~~~~~~~~~~~~~

\section{Submitting changes to the relax project}


% Submitting changes as a patch.
\subsection{Submitting changes as a patch}

The preferred method for submitting fixes and improvements to the relax source code is by the creation of a patch.  If your changes are a fix, make sure you have submitted a bug report to the bug tracker located at \href{https://gna.org/bugs/?group=relax}{https://gna.org/bugs/?group=relax} first.  See section~\ref{reporting bugs} on page~\pageref{reporting bugs} for more details.  Two methods can be used to generate the patch, either using the Unix command \texttt{diff} or using the Subversion program.  The resultant file \texttt{patch} of either the \texttt{diff} or \texttt{svn} command described below can be posted to the ``relax-devel at gna.org'' mailing list.  Please label within your post which version of relax you modified or which revision the patch is for.  Also try to create a commit log message according to the format described in section~\ref{commit log format} on page~\pageref{commit log format} for one of the relax committers to use as a template for committing the change.


% Modification of official releases -- creating patchs with diff.
\subsection{Modification of official releases -- creating patchs with diff}

If your modifications have been made to the source code of one of the official relax releases (for example 1.2.2), then the Unix command \texttt{diff} can be used to create a patch.  A patch file is simply the output of the diff command used recursively and presented in the `unified' format.  Therefore two directories need to be compared.  If the original sources are located in the directory \texttt{relax\_orig} and the modified sources in \texttt{relax\_mod}, then the patch can be created by typing

\example{\$ diff -ur relax\_orig relax\_mod > patch}


% Modification of the latest sources -- creating patchs with Subversion.
\subsection{Modification of the latest sources -- creating patchs with Subversion}

If possible, changes to the latest sources is preferred.  Using the most up to date sources from the relax SVN repository will significantly aid the relax developers to incorporate your changes back into the main development line.  To check out the current development line, see section~\ref{svn repository} on page~\pageref{svn repository} for details.  Prior to submitting a patch to the mailing list, your sources should be updated to include the most recent changes.  To do this, type

\example{\$ svn up}

and note the revision number to include in your post.  The update may cause a conflict if changes added to the repository clash with your modifications.  If this occurs, see the Subversion book at \href{http://svnbook.red-bean.com/}{http://svnbook.red-bean.com/} for details on how to resolve the conflict or submit a message to the relax-devel list.

Once the sources are up to date, your changes can be can be converted into the patch text file.  Using SVN, creating a patch is easy.  Just type

\example{\$ svn diff > patch}

in the base relax directory.



% Committers.
%~~~~~~~~~~~~

\section{Committers}


% Becoming a committer.
\subsection{Becoming a committer}\label{becoming a committer}

After proving oneself, anyone can become a relax developer and obtain commit access to the relax repository.  The main criteria for selection by the relax developers is to show good judgement, competence in producing good patches, compliance with the coding and commit log conventions, comportment on the mailing lists, not producing too many bugs, only taking on challenges which can be handled, and the skill in judging your own abilities.  You will also need to have an understanding of the concepts of version control, specifically those relating to Subversion.  The SVN book at \href{http://svnbook.red-bean.com/}{http://svnbook.red-bean.com/} contains all the version control information you will need.  After a number of patches have been submitted and accepted, any of the relax developers can propose that you receive commit access.  If a number of developers agree while no one says no, then commit access will be offered.

One area where coding ability can be demonstrated is within the relax test suite.  The addition of tests, especially those where the relax internal data structures of \texttt{self.relax.data} are scrutinised, can be a good starting point for learning the structure of relax.  The beauty of the tests is that the introduction of bugs has no effect on normal program execution.  The relax test suite is an ideal proving ground.

If skills in only certain areas of relax development, for example in creation of the documentation, an understanding of C but not python, an understanding of solely the code of the user interface, or an understanding of the code specific to a certain type of data analysis methodology, then partial commit access may be granted.  Although you will have the ability to make modifications to any part of the repository, please make modifications only those areas for which you have permission.


% Joining Gna!
\subsection{Joining Gna!}

The first step in becoming a committer is to create a Gna!\ account.  Go to \href{https://gna.org/account/register.php}{https://gna.org/account/register.php} and type in a login name, password, real name, and the email address you would like to use.  You will then get an automatic email from Gna!\ which will contain a link to activate your registration.


% Joining the relax project.
\subsection{Joining the relax project}

The second step in becoming a committer is to register to become a member of the relax project.  Go to the Gna!\ website (\href{https://gna.org/}{https://gna.org/}) and login.  Click on 'My Groups' to go to \href{https://gna.org/my/groups.php}{https://gna.org/my/groups.php}.  In the section 'Request for inclusion', type 'relax' and hit enter.  Select relax and write something in the comments.  If you have been approved (see section~\ref{becoming a committer}), then you will be added to the project.


% Format of the commit logs.
\subsection{Format of the commit logs}\label{commit log format}

If you are a relax developer and you have commit access to the repository, the following conventions should be followed for all commits.

The length of all lines in the commit log should never exceed 100 characters.  This is so that the log message viewed in either emails or by the command prompt command \mbox{\texttt{svn log}} is legible.  The first line of the commit log should be a short description or synopsis of the changes.  The second line should be blank.

If the commit is a bug fix reported by someone else or if the commit originates from a patch posted by someone else, the next lines should be reserved for crediting.  The name of the person and their obfuscated email address (for example edward at nmr-relax.com) should be included in the message.

If the commit relates to an entry in the bug tracker or to a discussion on the mailing lists, then the web address of either the bug report or the mailing list archive message should be cited in the next section (separated from the synopsis or credit section by a blank line).  All relevant links should be included to allow easy navigation between the repository, mailing lists, bug tracker, etc.  An example is bug \#5559 which is located at \href{https://gna.org/bugs/?func=detailitem\&item\_id=5559}{https://gna.org/bugs/?func=detailitem\&item\_id=5559} and the post to ``relax-devel at gna.org'' describing the fix to that bug which is located at \href{https://mail.gna.org/public/relax-devel/2006-03/msg00013.html}{https://mail.gna.org/public/relax-devel/2006-03/msg00013.html}.

A full description containing all the details can follow.  This description should follow a blank line, can itself be sectioned using blank lines, and finally the log is terminated by one blank line at the end of the message.



% The Sconstruct build system.
%~~~~~~~~~~~~~~~~~~~~~~~~~~~~~

\section{The Sconstruct build system}

\index{Sconstruct|textbf}
The Sconstruct build system was chosen over other build systems including `make' as it is a cross-platform build system which can be used in Unix, GNU/Linux, Mac OS X, and even Windows (the correct compilers are nevertheless required).  Various components of the program relax can be created using the Sconstruct utility.  This includes C module compilation, manual creation, distribution creation, and cleaning up and removing certain files.  The file `sconstruct' in the base relax directory, which consists of python code, directs the operation of Sconstruct for the various functions.


% C module compilation.
\subsection{C module compilation}

\index{C module compilation|textbf}
As described in the installation chapter, typing \texttt{`scons'} in the base directory will create the shared objects which are imported into Python as modules.


% Creation of the PDF manual.
\subsection{Creation of the PDF manual}

\index{manual!PDF creation|textbf}
To create the PDF version of the relax manual, type

\example{\$ scons manual}

in the base directory.  Sconstruct will then run a series of shell commands to create the manual from the \LaTeX\ sources located in the \texttt{`docs/latex'} directory.  This is dependent on the programs \texttt{`latex'}, \texttt{`makeindex'}, \texttt{`dvips'}, and \texttt{`ps2pdf'} being located within the environment's path.


% Creation of the HTML manual.
\subsection{Creation of the HTML manual}

\index{manual!HTML creation|textbf}
The HTML version of the relax manual is made by typing

\example{\$ scons manual\_html}

in the base directory.  One command calling the program \texttt{`latex2html'} will be executed which will create HTML pages from the \LaTeX\ sources.


% Making distribution archives.
\subsection{Making distribution archives}

\index{distribution|textbf}
Two types of distribution archive can be created from the currently checked out sources, the source and binary distributions.  To create the source distribution, type 

\example{\$ scons source\_dist}

while to create the binary distribution, whereby the C modules are compiled and the resultant shared objects are included in the bzipped tar file, type

\example{\$ scons binary\_dist}

If a binary distribution does not exist for your architecture, feel free to create it yourself and contribute the archive to be included on the download pages.  To do this, you will need to check out the appropriate tagged branch from the relax subversion repository.  If the current stable release is called 1.2.3, then check out that branch by typing

\example{\$ svn co svn+ssh://bugman@svn.gna.org/svn/relax/tags/1.2.3 relax}

replacing `bugman' with your user name if you are a relax developer, or typing

\example{\$ svn co svn://svn.gna.org/svn/relax/tags/1.2.3 relax}

otherwise.  Then build the binary distribution and send a message to the relax development mailing list.  If compilation does not work, please submit a bug to the bug tracker system at \href{https://gna.org/bugs/?group=relax}{https://gna.org/bugs/?group=relax} detailing the relax version, operation system, architecture, and any other information you believe will help to solve the problem.  More information about donating binary distributions to the relax project is given in the open source infrastructure chapter.


% Cleaning up.
\subsection{Cleaning up}

\index{clean up|textbf}
If the command

\example{\$ scons clean}

is run in the base directory, all Python byte compiled files `*.pyc', all C object files `*.o' and `*.os', all C shared object files `*.so', and any backup files with the extension `*.bak' are removed from all sub-directories.  In addition, any temporary \LaTeX\ compilation files are removed from the \texttt{`docs/latex'} directory.
