% Development of relax chapter.
%%%%%%%%%%%%%%%%%%%%%%%%%%%%%%%

\chapter{Development of relax}

This chapter is written for developers or those who would like to extend the functionality of relax.  It is not required for using relax.



% Coding conventions.
%~~~~~~~~~~~~~~~~~~~~

\section{Coding conventions}


% Indentation.
\subsection{Indentation}

Indentation should be set to four spaces rather than a tab character.  This is the recommendation given in the python style guide found at \texttt{`http://www.python.org/doc/essays/styleguide.html'}.  For vi, add the following lines to \texttt{`\~/.vimrc'}:

\begin{exampleenv}
set tabstop=4 \\
set shiftwidth=4 \\
set expandtab
\end{exampleenv}

Certain versions of vim, those within the 6.2 series, contain a bug where the tabstop value cannot be changed using the \texttt{`\~/.vimrc'} file (although typing \texttt{`:set tabstop=4'} in vim will fix it).  One solution is to edit the file \texttt{`python.vim'} which is located in the path \texttt{`/usr/share/vim/ftplugin/'} or equivalent. It contains the two lines:

\begin{exampleenv}
" Python always uses a `tabstop' of 8. \\
setlocal ts=8
\end{exampleenv}

If these lines are deleted, the bug will be removed.  Another way to fix the problem is to install newer versions of the run-time files (which will pretty much do the same thing).


% Doc strings.
\subsection{Doc strings}

These should be set to no more than 100 characters long including all leading white space.  The standard Python convention of a one line description separated from a detailed description by an empty line should be adhered to.



% The make system.
%~~~~~~~~~~~~~~~~~

\section{The make system}

For UNIX like systems, including GNU/Linux, various components of the program relax can be created using the Unix make utility.  This includes C module compilation, manual creation, distribution creation, and cleaning up and removing certain files.  Makefiles exist in the various directories to facilitate this process.


% C module compilation.
\subsection{C module compilation}

As described in the installation chapter, typing \texttt{`make'} in the base directory will create the shared objects which are imported into Python as modules.


% Creation of the PDF manual.
\subsection{Creation of the PDF manual}

To create the PDF version of the relax manual, type

\example{make manual}

in the base directory.  Make will then shift to the \texttt{`docs/latex'} directory and run a series of shell commands to create the manual from the \LaTeX\ sources.  This is dependent on the programs \texttt{`rm'}, \texttt{`latex'}, \texttt{`makeindex'}, \texttt{`dvips'}, and \texttt{`ps2pdf'} being located in the environment's path.


% Creation of the HTML manual.
\subsection{Creation of the HTML manual}

The HTML version of the relax manual is made by typing

\example{make manual\_html}

in the base directory.  Again make will then shift to the \texttt{`docs/latex'} directory.  One command calling the program \texttt{`latex2html'} will be executed which will create HTML pages from the \LaTeX\ sources.


% Making distribution archives.
\subsection{Making distribution archives}

This section is incomplete.
Creation of the HTML manual


% Cleaning up.
\subsection{Cleaning up}

If the command

\example{make clean}

is run in the base directory, all Python byte compiled files `*.pyc', all C object files `*.o', all C shared object files `*.so', and any backup files with the extension `*.bak' are removed from all subdirectoies.  In addition, any temporary \LaTeX\ compilation files are removed from the \texttt{`docs/latex'} directory.
