% Development of relax chapter.
%%%%%%%%%%%%%%%%%%%%%%%%%%%%%%%

\chapter{Development of relax}

This chapter is written for developers or those who would like to extend the functionality of relax.  It is not required for using relax.  If you would like to modify relax to suit your needs please subscribe to all the relax mailing lists\index{mailing list} (see the open source infrastructure chapter for more details).  Announcements are sent to ``relax-announce at gna.org''\index{mailing list!relax-announce} whereas ``relax-users at gna.org''\index{mailing list!relax-users} is the list where discussions about the usage of relax should be posted.  ``relax-devel at gna.org''\index{mailing list!relax-devel} is where all discussions about the development of relax including feature requests, program design, or any other discussions relating to relax's structure or code should be posted.  Finally, ``relax-commits at gna.org''\index{mailing list!relax-commits} is where all changes to relax's code and documentation, as well as changes to the web pages, are automatically sent to.  Anyone interested in joining the project should subscribe to all four lists.



% Version control using Subversion.
%~~~~~~~~~~~~~~~~~~~~~~~~~~~~~~~~~~

\section{Version control using Subversion}\label{svn repository}

The development of relax requires the use of the Subversion\index{Subversion|textbf} (SVN)\index{SVN|textbf} version control software \href{http://subversion.tigris.org/}{http://subversion.tigris.org/}.  The source code to relax is stored in an SVN repository located at \href{http://svn.gna.org/svn/relax/}{http://svn.gna.org/svn/relax/}.  Every single change ever made to the program is recorded in this repository\index{repository}.  For more information see the open source infrastructure chapter.

Although the downloadable distribution archives\index{distribution archive} can be modified it is best that the most current and up to date revision (the \textit{head} revision) is modified instead.  More information about the basics of version control and how this is implemented in Subversion can be found in the Subversion book located at \href{http://svnbook.red-bean.com/}{http://svnbook.red-bean.com/}.

If you are not currently a relax developer you can check out the head revision, assuming that 1.2 is the current major version number, by typing

\example{\$ svn co svn://svn.gna.org/svn/relax/1.2 relax}

Otherwise if you are a developer type

\example{\$ svn co svn+ssh://xxxxx@svn.gna.org/svn/relax/1.2 relax}

replacing \texttt{xxxxx} with your Gna!\ login name.  If your version is out of date it can be updated to the latest revision by typing

\example{\$ svn up}

Modifications can be made to these sources.



% Coding conventions.
%~~~~~~~~~~~~~~~~~~~~

\section{Coding conventions}

The following conventions must be followed at all times for any code to be accepted into the relax repository.



% Indentation.
\subsection{Indentation}

\index{indentation|textbf}
Indentation should be set to four spaces rather than a tab character.  This is the recommendation given in the python style guide found at \href{http://www.python.org/doc/essays/styleguide.html}{http://www.python.org/doc/essays/styleguide.html}.  Emacs should automatically set the tabstop correctly.  For vi add the following lines to the \texttt{vimrc} file:

\begin{exampleenv}
set tabstop=4 \\
set shiftwidth=4 \\
set expandtab
\end{exampleenv}

For UNIX systems, including Linux and Mac OS X, the \texttt{vimrc} file is \texttt{`$\sim$/.vimrc'} whereas in MS Windows the file is \texttt{`\$VIM/\_vimrc'} which is usually \texttt{`C:$\backslash$Program Files$\backslash$vim$\backslash$\_vimrc'}.  Certain versions of vim, those within the 6.2 series, contain a bug where the tabstop value cannot be changed using the \texttt{vimrc} file (although typing \texttt{`:set tabstop=4'} in vim will fix it).  One solution is to edit the file \texttt{`python.vim'} which on GNU/Linux systems is located in the path \texttt{`/usr/share/vim/ftplugin/'}.  It contains the two lines

\begin{exampleenv}
" Python always uses a `tabstop' of 8. \\
setlocal ts=8
\end{exampleenv}

If these lines are deleted the bug will be removed.  Another way to fix the problem is to install newer versions of the run-time files (which will do the same thing).



% Doc strings.
\subsection{Doc strings}
\index{doc string|textbf}

These should be set to no more than 100 characters long including all leading white space.  The standard Python convention of a one line description separated from a detailed description by an empty line should be adhered to.  All functions should have a docstring describing in detail the structure and organisation of the code.



% Variable, function, and class names.
\subsection{Variable, function, and class names}

In relax a mixture of both camel case (eg. CamelCase)\index{camel case} and lower case with underscores is used.  Despite the variability there are fixed rules which should be adhered to.  These naming conventions should be observed at all times.



% Variables and functions.
\subsubsection{Variables and functions}

For both variables and functions lower case with underscores between words is always used.  This is for readability as the convention is much more fluent than camel case.  A few rare exceptions exist, an example is the Brownian diffusion tensor parameter of anisotropy $\Diff_a$ which is referenced as \texttt{self.relax.data.diff[run].Da}.  As a rule though all new variable or function names should be kept as lower case.



% Classes.
\subsubsection{Classes}

For classes relax uses a mix of camel case (for example all the \texttt{RelaxError} objects) and underscores (for example \texttt{Model\_free}).  The first letter in all cases is always capitalised.  Generally the camel case is reserved for very low level classes which are involved in the program's infrastructure.  Examples include the RelaxError code, the threading code, and the \texttt{self.relax.data} code.  All the data analysis specific code, generic code, interface code, etc.\ uses underscores between the words with only the first letter capitalised.  One exception is the space mapping class \texttt{OpenDX}, the reason being that the program is called `OpenDX'.



% Submitting changes to the relax project.
%~~~~~~~~~~~~~~~~~~~~~~~~~~~~~~~~~~~~~~~~~

\section{Submitting changes to the relax project}


% Submitting changes as a patch.
\subsection{Submitting changes as a patch}
\index{patch|textbf}

The preferred method for submitting fixes and improvements to the relax source code is by the creation of a patch.  If your changes are a fix make sure you have submitted a bug report\index{bug report} to the bug tracker\index{bug tracker} located at \href{https://gna.org/bugs/?group=relax}{https://gna.org/bugs/?group=relax} first.  See section~\ref{reporting bugs} on page~\pageref{reporting bugs} for more details.  Two methods can be used to generate the patch -- the Unix command \texttt{diff} or the Subversion program.  The resultant file \texttt{patch} of either the \texttt{diff} or \texttt{svn} command described below can be posted to the ``relax-devel at gna.org'' mailing list\index{mailing list!relax-devel}.  Please label within your post which version of relax you modified or which revision the patch is for.  Also try to create a commit log\index{commit log} message according to the format described in section~\ref{commit log format} on page~\pageref{commit log format} for one of the relax committers to use as a template for committing the change.



% Modification of official releases -- creating patches with diff.
\subsection{Modification of official releases -- creating patches with diff}
\index{patch!diff|textbf}

If your modifications have been made to the source code of one of the official relax releases (for example 1.2.2) then the Unix command \texttt{diff} can be used to create a patch.  A patch file is simply the output of the diff command run with the recursive flag and presented in the `unified' format.  Therefore two directories need to be compared.  If the original sources are located in the directory \texttt{relax\_orig} and the modified sources in \texttt{relax\_mod} then the patch can be created by typing

\example{\$ diff -ur relax\_orig relax\_mod > patch}



% Modification of the latest sources -- creating patches with Subversion.
\subsection{Modification of the latest sources -- creating patches with Subversion}
\index{patch!Subversion|textbf}
\index{Subversion!patch|textbf}

If possible changes to the latest sources is preferred.  Using the most up to date sources from the relax SVN repository will significantly aid the relax developers to incorporate your changes back into the main development line.  To check out the current development line see section~\ref{svn repository} on page~\pageref{svn repository} for details.  Prior to submitting a patch to the mailing list your sources should be updated to include the most recent changes.  To do this type

\example{\$ svn up}

and note the revision number to include in your post.  The update may cause a conflict if changes added to the repository clash with your modifications.  If this occurs see the Subversion book at \href{http://svnbook.red-bean.com/}{http://svnbook.red-bean.com/} for details on how to resolve the conflict or submit a message to the rela\mbox{x-d}evel list\index{mailing list!relax-devel}.

Once the sources are up to date your changes can be can be converted into the patch text file.  Using SVN creating a patch is easy.  Just type

\example{\$ svn diff > patch}

in the base relax directory.



% Committers.
%~~~~~~~~~~~~

\section{Committers}


% Becoming a committer.
\subsection{Becoming a committer}\label{becoming a committer}

Anyone can become a relax developer and obtain commit access\index{commit access} to the relax repository.  The main criteria for selection by the relax developers is to show good judgement, competence in producing good patches, compliance with the coding and commit log conventions, comportment on the mailing lists, not producing too many bugs, only taking on challenges which can be handled, and the skill in judging your own abilities.  You will also need to have an understanding of the concepts of version control specifically those relating to Subversion\index{Subversion}.  The SVN book at \href{http://svnbook.red-bean.com/}{http://svnbook.red-bean.com/} contains all the version control information you will need.  After a number of patches have been submitted and accepted any of the relax developers can propose that you receive commit access.  If a number of developers agree while no one says no then commit access will be offered.

One area where coding ability can be demonstrated is within the relax test suite\index{test suite}.  The addition of tests, especially those where the relax internal data structures of \texttt{self.relax.data} are scrutinised, can be a good starting point for learning the structure of relax.  This is because the introduction of bugs has no effect on normal program execution.  The relax test suite is an ideal proving ground.

If skills in only certain areas of relax development, for example in creation of the documentation, an understanding of C but not python, an understanding of solely the code of the user interface, or an understanding of the code specific to a certain type of data analysis methodology, then partial commit access may be granted.  Although you will have the ability to make modifications to any part of the repository please make modifications only those areas for which you have permission.



% Joining Gna!
\subsection{Joining Gna!}
\index{Gna}

The first step in becoming a committer is to create a Gna!\ account.  Go to \href{https://gna.org/account/register.php}{https://gna.org/account/register.php} and type in a login name, password, real name, and the email address you would like to use.  You will then get an automatic email from Gna!\ which will contain a link to activate your registration.



% Joining the relax project.
\subsection{Joining the relax project}

The second step in becoming a committer is to register to become a member of the relax project.  Go to the Gna!\ website\index{Gna} (\href{https://gna.org/}{https://gna.org/}) and login.  Click on `My Groups' to go to \href{https://gna.org/my/groups.php}{https://gna.org/my/groups.php}.  In the section `Request for inclusion' type `relax' and hit enter.  Select relax and write something in the comments.  If you have been approved (see section~\ref{becoming a committer}) you will be added to the project committers list.



% Format of the commit logs.
\subsection{Format of the commit logs}\label{commit log format}
\index{commit log|textbf}

If you are a relax developer and you have commit access to the repository the following conventions should be followed for all commit messages.

The length of all lines in the commit log should never exceed 100 characters.  This is so that the log message viewed in either emails or by the command prompt command \mbox{\texttt{svn log}} is legible.  The first line of the commit log should be a short description or synopsis of the changes.  The second line should be blank.

If the commit is a bug fix reported by someone else or if the commit originates from a patch posted by someone else the next lines should be reserved for crediting.  The name of the person and their obfuscated email address (for example edward at nmr-relax.com) should be included in the message.

If the commit relates to an entry in the bug tracker or to a discussion on the mailing lists then the web address of either the bug report or the mailing list archive message should be cited in the next section (separated from the synopsis or credit section by a blank line).  All relevant links should be included to allow easy navigation between the repository\index{repository}, mailing lists\index{mailing list}, bug tracker\index{bug tracker}, etc.  An example is bug \#5559 which is located at \href{https://gna.org/bugs/?func=detailitem\&item\_id=5559}{https://gna.org/bugs/?func=detailitem\&item\_id=5559} and the post to ``relax-devel at gna.org''\index{mailing list!relax-devel} describing the fix to that bug which is located at \href{https://mail.gna.org/public/relax-devel/2006-03/msg00013.html}{https://mail.gna.org/public/relax-devel/2006-03/msg00013.html}.

A full description with all the details can follow.  This description should follow a blank line, can itself be sectioned using blank lines, and finally the log is terminated by one blank line at the end of the message.



% Discussing major changes.
\subsection{Discussing major changes}

If you are contemplating major changes, either for a bug fix, to add a completely new feature or user function for your own work, to improve the behaviour of part the program, or any other potentially disruptive modifications, please discuss these ideas on the rela\mbox{x-d}evel mailing list\index{mailing list!relax-devel}.  If the planned changes have the potential to introduce problems the creation of a private branch may be suggested.



% Branches.
\subsection{Branches}
\index{branches|textbf}
\index{repository!branches|textbf}

If a change is likely to be disruptive or cause breakages in the program the use of your own temporary branch is recommended.  This private branch is a complete copy of one of the main development lines wherein you can make changes without disrupting the other developers.  Although called a private branch every change is visible to all other developers and each commit will result in an automatic email to the relax-commits mailing list\index{mailing list!relax-commits}.  Other developers are even able to check out your branch and make modifications to it.  Private branches can also be used for testing ideas.  If the idea does not work the branch can be deleted from the repository (in reality the branch will always exist between the revision numbers of its creation and deletion and can always be resurrected).  For example to create a branch from the main 1.2 development line called \texttt{molmol\_macros} whereby new Molmol macros are to be written, type

\begin{exampleenv}
\$ svn cp svn+ssh://xxxxx@svn.gna.org/svn/relax/1.2 $\backslash$ \\
 svn+ssh://xxxxx@svn.gna.org/svn/relax/branches/molmol\_macros
\end{exampleenv}

replacing \texttt{xxxxx} with your login name.  You can then check out your private branch by typing

\example{\$ svn co svn+ssh://xxxxx@svn.gna.org/svn/relax/branches/molmol\_macros}

which will create a directory called \texttt{molmol\_macros} containing all the relax source files.  To have the files dumped into a different directory type the name of that directory at the end of the last command.  Modifications can be made to this copy while normal development continues on the main line.  Once the desired changes have been made and reviewed the changes which have occurred on the main line can be merged into your branch\index{merge}\index{repository!merge}.  If development is taking a long time then merging should occur on a regular basis to avoid large incompatible changes forming between the two branches.  For example to merge the changes which have occurred between the initial branch, say r2351 (revision number 2351), and r2378 of the main development line type in the base directory of your branch

\example{\$ svn merge -r2351:2378 svn+ssh://xxxxx@svn.gna.org/svn/relax/1.2 .}
\index{Subversion!merge}

The differences will have been merged into your checked out copy.  If conflicts\index{Subversion!conflict} have occurred see the Subversion book at \href{http://svnbook.red-bean.com/}{http://svnbook.red-bean.com/} for information on how to resolve the problem.  Otherwise the changes to your branch can be committed:

\example{\$ svn ci}
\index{Subversion!commit}

Make sure to include in your commit message the revision numbers which have been merged (cutting and pasting the command would be useful).  This is important because if new changes need to be merged again, for example up to r2401, then you will need to type

\example{\$ svn merge -r2378:2401 svn+ssh://xxxxx@svn.gna.org/svn/relax/1.2 .}
\index{Subversion!merge}

Note that the changes from r2351 to r2378 have not been merged for a second time.  This is important and is the reason that the revision numbers need to be noted in the commit logs.  Once you have completed your modifications, you have merged all changes which have occurred in the main line, and the changes have been approved for merging back into the main line then your branch can be merged.  First check out a copy of the main line,

\example{\$ svn co svn+ssh://xxxxx@svn.gna.org/svn/relax/1.2 relax}
\index{Subversion!check out}

or update a previously checked out version,

\example{\$ svn up}
\index{Subversion!update}

Assuming the initial branch occurred at r2351 and the final modification occurred at r2430 then in the base directory of the checked out main line type

\example{\$ svn merge -r2351:2430 svn+ssh://xxxxx@svn.gna.org/svn/relax/branches/molmol\_macros .}
\index{Subversion!merge}

and then check in the modifications.  Your changes will now be present in the main line.  The last step is to delete your private branch

\example{\$ svn rm svn+ssh://xxxxx@svn.gna.org/svn/relax/branches/molmol\_macros}
\index{Subversion!remove}



% The Sconstruct build system.
%~~~~~~~~~~~~~~~~~~~~~~~~~~~~~

\section{The Sconstruct build system}

\index{Sconstruct|textbf}
The Sconstruct build system was chosen over other build systems including \texttt{`make'}\index{make} as it is a cross-platform build system which can be used in Unix\index{Unix}, GNU/Linux\index{GNU/Linux}, Mac OS X\index{Mac OS X}, and even MS Windows\index{MS Windows} (the correct compilers are nevertheless required).  Various components of the program relax can be created using the Sconstruct utility.  This includes C module compilation, manual creation, distribution creation, and cleaning up and removing certain files.  The file `sconstruct' in the base relax directory, which consists of python code, directs the operation of Sconstruct for the various functions.



% C module compilation.
\subsection{C module compilation}
\index{C module compilation|textbf}
\index{Sconstruct!C module compilation|textbf}

As described in the installation chapter typing \texttt{`scons'} in the base directory will create the shared objects which are imported into Python as modules.



% Creation of the PDF manual.
\subsection{Creation of the PDF manual}
\index{manual!PDF creation|textbf}

To create the PDF version of the relax manual type

\example{\$ scons manual}
\index{Sconstruct!PDF manual|textbf}

in the base directory.  Sconstruct will then run a series of shell commands to create the manual from the \LaTeX\ sources located in the \texttt{`docs/latex'} directory.  This is dependent on the programs \texttt{`latex'}, \texttt{`makeindex'}, \texttt{`dvips'}, and \texttt{`ps2pdf'} being located within the environment's path.



% Creation of the HTML manual.
\subsection{Creation of the HTML manual}
\index{manual!HTML creation|textbf}

The HTML version of the relax manual is made by typing

\example{\$ scons manual\_html}
\index{Sconstruct!HTML manual|textbf}

in the base directory.  One command calling the program \texttt{`latex2html'} will be executed which will create HTML pages from the \LaTeX\ sources.



% Making distribution archives.
\subsection{Making distribution archives}
\index{distribution archive|textbf}

Two types of distribution archive can be created from the currently checked out sources -- the source and binary distributions.  To create the source distribution type 

\example{\$ scons source\_dist}
\index{Sconstruct!source distribution|textbf}

whereas to create the binary distribution, whereby the C modules are compiled and the resultant shared objects are included in the bzipped tar file, type

\example{\$ scons binary\_dist}
\index{Sconstruct!binary distribution|textbf}

If a binary distribution does not exist for your architecture feel free to create it yourself and contribute the archive to be included on the download pages.  To do this you will need to check out the appropriate tagged branch from the relax subversion repository.  If the current stable release is called 1.2.3 check out that branch by typing

\example{\$ svn co svn+ssh://bugman@svn.gna.org/svn/relax/tags/1.2.3 relax}
\index{Subversion!check out}

replacing `bugman' with your user name if you are a relax developer, otherwise typing

\example{\$ svn co svn://svn.gna.org/svn/relax/tags/1.2.3 relax}
\index{Subversion!check out}

Then build the binary distribution and send a message to the relax development mailing list\index{mailing list!relax-devel}.  If compilation does not work please submit a bug to the bug tracker\index{bug tracker} system at \href{https://gna.org/bugs/?group=relax}{https://gna.org/bugs/?group=relax} detailing the relax version, operation system, architecture, and any other information you believe will help to solve the problem.  More information about donating binary distributions to the relax project is given in the open source infrastructure chapter.



% Cleaning up.
\subsection{Cleaning up}
\index{clean up|textbf}

If the command

\example{\$ scons clean}
\index{Sconstruct!clean up|textbf}

is run in the base directory all Python byte compiled files \texttt{`*.pyc'}, all C object files \texttt{`*.o'} and \texttt{`*.os'}, all C shared object files \texttt{`*.so'}, and any backup files with the extension \texttt{`*.bak'} are removed from all sub-directories.  In addition any temporary \LaTeX\ compilation files are removed from the \texttt{`docs/latex'} directory.



% The core design of relax.
%~~~~~~~~~~~~~~~~~~~~~~~~~~

\section{The core design of relax}

To enable flexibility the internal structure of relax is modular.  By construction the large collection of independent data analysis tools can be used individually and relatively easily by any new code implementing other forms of relaxation data analysis or even by other programs.  The core modular design of the program is shown in Figure~\ref{fig: core design}.



% The divisions of relax's source code.
\subsection{The divisions of relax's source code}

relax's source code can be divided into five major areas:  the initialisation code, the user interface (UI) code, the functional code, the number crunching code, and the code storing the program state.

\begin{description}
\item[Initialisation:]  The code belonging to this section initialises the program, processes the command-line arguments, and determines what mode the program will be run in including the choice of the UI.

\item[UI:]  The user interface\index{UI|textbf}.  Currently the prompt and the script are the only user interfaces into relax.  There are other program modes which are not part of a user interface.  These include the test mode in which the program instantly exits and threading mode which is spawned by a parent process and waits for commands.  In the future a graphical user interface (GUI), a web based interface, or any other type of interface may be added.

\item[Functional code:]  This code is the main part of the program.  It includes anything which does not fit into the other sections and comprises the generic code, the specific code, and the specific setup code used as an interface between the two.

\item[Number crunching:]  The computationally expensive code belongs in this section.

\item[Program state:]  The state of the program is contained within the data structure \texttt{self.relax.data} which is accessible from all parts of the program.  It should only be read by the generic, specific, and number crunching code.  Only the generic and specific code should change its contents.
\end{description}

% Core relax design figure.
\begin{figure}
\centerline{\includegraphics[width=\textwidth, bb=57 274 447 750]{images/core_design.eps.gz}}
\caption{The core design of relax} \label{fig: core design}
\end{figure}



% The major components of relax.
\subsection{The major components of relax}

Each of the boxes in Figure~\ref{fig: core design} represents a different grouping of code.
\begin{description}
\item[relax:]  The top level module.  This initialises the entire program, tests the dependencies, places the custom errors into the module \texttt{\_\_builtin\_\_}, and prints the program's introduction message.

\item[Command line arguments:]  This code processes the arguments supplied to the program and decides whether to print the help message, initialise the prompt, execute a script, initialise a different UI, run the program in test mode, or run the program as a slave thread.

\item[Prompt:]  The main user interface consisting of a Python\index{Python} prompt\index{prompt|textbf}.  The namespace of the interpreter contains the various user functions which are front ends to the generic code.  The user functions are simply Python functions which test the supplied arguments to make sure they are of the correct type (string\index{string}, integer\index{integer}, list\index{list}, or any other type) before sending the values to the generic code.  The code for the prompt is located in the directory \texttt{prompt/}.

\item[Script:]  If a script\index{script|textbf} is supplied on the command line or executed within another user interface it will be run in the same namespace as that of the prompt.  Hence the script has access to all the user functions available at the relax prompt.  This allows commands which are typed at the prompt to be pasted directly and unmodified into a text file to be run as a script.

\item[GUI:]  The graphical user interface\index{GUI|textbf}.  Although not coded the most mature and least destabilising widget set to use would be QT\index{QT}.  The GUI should be relatively easy to tie into relax.  The design is such that the GUI can be dropped straight into relax without effecting the normal prompt and script based operation of the program.

\item[Other interfaces:]  Any number of interfaces for example other GUIs, an ncurses interface, a web based interface, or an MPI interface could be added to relax without modification of the current sources.

\item[Generic code:]  This code includes classes and functions which are independent of the UI and not specific to a certain run type, for example not being involved in model-free analysis, relaxation curve-fitting, the NOE calculation, and reduced spectral density mapping.  All this code is located in the directory \texttt{generic\_fns/}.

\item[Specific setup:]  This code implements the internal interface between the generic and specific code.  The generic code calls the specific setup asking for a specific function for the given run type.  For example by asking for the minimise function when the run type is model-free analysis the function \texttt{self.rel\-ax.spec\-if\-ic.mod\-el\_free.min\-im\-ise()} is returned.  Although the generic code accesses the specific code solely through this interface the specific code can access the generic code directly.  The code is located in the file \texttt{specific\_fns/specific\_setup.py}.

\item[Specific code:]  This is the code which is specific to the run type -- model-free analysis, relaxation curve-fitting, reduced spectral density mapping, and the NOE calculation.  Each type is located in a separate file in the directory \texttt{specific\_fns/}.

\item[Mathematical functions:]  This is reserved for CPU intensive code involved in calculations.  The code may be written in Python however C code can be used to significantly increase the speed of the calculations.  For optimisation the code can include function evaluations, calculation of gradients, and calculation of Hessians.  These functions are located in the directory \texttt{maths\_fns/}.

\item[Data:]  The program state stored in the class \texttt{self.relax.data}.  This class contains all the program data and is accessed by the generic and specific code.  The mathematical functions may also access this data but this is not recommended.  The structure is initialised by the file \texttt{data.py} and the data is modified solely by the generic and specific code.
\end{description}
