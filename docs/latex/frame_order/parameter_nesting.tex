%%%%%%%%%%%%%%%%%%%%%%%%%%%%%%%%%%%%%%%%%%%%%%%%%%%%%%%%%%%%%%%%%%%%%%%%%%%%%%%
%                                                                             %
% Copyright (C) 2015,2017 Edward d'Auvergne                                   %
%                                                                             %
% This file is part of the program relax (http://www.nmr-relax.com).          %
%                                                                             %
% This program is free software: you can redistribute it and/or modify        %
% it under the terms of the GNU General Public License as published by        %
% the Free Software Foundation, either version 3 of the License, or           %
% (at your option) any later version.                                         %
%                                                                             %
% This program is distributed in the hope that it will be useful,             %
% but WITHOUT ANY WARRANTY; without even the implied warranty of              %
% MERCHANTABILITY or FITNESS FOR A PARTICULAR PURPOSE.  See the               %
% GNU General Public License for more details.                                %
%                                                                             %
% You should have received a copy of the GNU General Public License           %
% along with this program.  If not, see <http://www.gnu.org/licenses/>.       %
%                                                                             %
%%%%%%%%%%%%%%%%%%%%%%%%%%%%%%%%%%%%%%%%%%%%%%%%%%%%%%%%%%%%%%%%%%%%%%%%%%%%%%%


\documentclass[varwidth=100cm, border=1pt]{standalone}

% Good looking tables.
\usepackage{booktabs}

% Better maths.
\usepackage{amsmath}
\usepackage{amssymb}

% Colours.
\usepackage[svgnames]{xcolor}

% Graphics for tables (after xcolor because of package clashing).
\usepackage{tikz}
\usetikzlibrary{calc}
\newcommand*\sethighlight[1]{\tikz[baseline=(char.base)]{\node[shape=rectangle,draw=blue,inner sep=2pt] (char) {#1};}}
\newcommand{\tikzmark}[1]{\tikz[overlay,remember picture] \node (#1) {};}

% Maths commands - frame order specific.
\newcommand{\Eigenseta}{\mathfrak{E}^\alpha_{\textrm{ax}}}
\newcommand{\Eigensetabc}{\mathfrak{E}_{\alpha\beta\gamma}}
\newcommand{\Eigensetax}{\mathfrak{E}_{\textrm{ax}}}
\newcommand{\Pivotsetone}{\mathfrak{p}_1}
\newcommand{\Pivotsettwo}{\mathfrak{p}_2}
\newcommand{\Posset}{\mathfrak{P}}
\newcommand{\Possetred}{\mathfrak{P}'}
\newcommand{\conesmax}{\sigma_{\textrm{max}}}
\newcommand{\conesmaxtwo}{\sigma_{\textrm{max,2}}}
\newcommand{\conetheta}{\theta}
\newcommand{\conethetax}{\theta_x}
\newcommand{\conethetay}{\theta_y}



\begin{document}

\begin{table}
\begin{tabular}{lccccccccc}

% Spacing.
\\[-5pt]

% Header.
\toprule
Model & \multicolumn{4}{l}{Parameter sets} & \multicolumn{4}{l}{Order parameters} & Grid search \\
      & & & & & & & & & dimensionality \\
\midrule

Rigid &
\tikzmark{P1}\sethighlight{$\Posset$} &
- &
- &
- &
- &
- &
- &
- \vphantom{\sethighlight{$\conesmaxtwo$}} &
6 (6) \\

Rotor &
\tikzmark{P2}$\Posset$ &
\tikzmark{E2}\sethighlight{$\Eigenseta$} &
\tikzmark{p2}\sethighlight{$\Pivotsetone$} &
- &
- &
- &
\sethighlight{$\conesmax$}\tikzmark{s2} &
- \vphantom{\sethighlight{$\conesmaxtwo$}} &
5 (11) \\

Isotropic cone &
\tikzmark{P3}$\Posset$ &
\tikzmark{E3}$\Eigensetax$\tikzmark{E3b} &
\tikzmark{p3}$\Pivotsetone$ &
- &
\tikzmark{S3}\sethighlight{$\conetheta$} &
- &
$\conesmax$\tikzmark{s3} &
- \vphantom{\sethighlight{$\conesmaxtwo$}} &
1 (13) \\

Pseudo-ellipse &
\tikzmark{P4}$\Posset$ &
\tikzmark{E4}\sethighlight{$\Eigensetabc$} &
\tikzmark{p4}$\Pivotsetone$ &
- &
\tikzmark{S4}$\conethetax$\tikzmark{S4b} &
\sethighlight{$\conethetay$}\tikzmark{Sy4} &
$\conesmax$\tikzmark{s4} &
- \vphantom{\sethighlight{$\conesmaxtwo$}} &
4 (15) \\

Isotropic cone, torsionless &
\tikzmark{P5}$\Posset$ &
$\Eigensetax$\tikzmark{E5} &
\tikzmark{p5}$\Pivotsetone$ &
- &
\tikzmark{S5}$\conetheta$ &
- &
- &
- \vphantom{\sethighlight{$\conesmaxtwo$}} &
0 (12) \\

Pseudo-ellipse, torsionless &
\tikzmark{P6}$\Posset$ &
\tikzmark{E6}$\Eigensetabc$ &
\tikzmark{p6}$\Pivotsetone$ &
- &
$\conethetax$\tikzmark{S6} &
$\conethetay$\tikzmark{Sy6} &
- &
- \vphantom{\sethighlight{$\conesmaxtwo$}} &
0 (14) \\

Free rotor &
\sethighlight{$\Possetred$}\tikzmark{P7} &
\tikzmark{E7}$\Eigenseta$ &
\tikzmark{p7}$\Pivotsetone$ &
- &
- &
- &
- &
- \vphantom{\sethighlight{$\conesmaxtwo$}} &
5 (9) \\

Isotropic cone, free rotor &
$\Possetred$\tikzmark{P8} &
$\Eigensetax$\tikzmark{E8} &
\tikzmark{p8}$\Pivotsetone$ &
- &
\tikzmark{S8}$\conetheta$ &
- &
- &
- \vphantom{\sethighlight{$\conesmaxtwo$}} &
0 (11) \\

Pseudo-ellipse, free rotor &
$\Possetred$\tikzmark{P9} &
\tikzmark{E9}$\Eigensetabc$ &
\tikzmark{p9}$\Pivotsetone$ &
- &
$\conethetax$\tikzmark{S9} &
$\conethetay$\tikzmark{Sy9} &
- &
- \vphantom{\sethighlight{$\conesmaxtwo$}} &
0 (13) \\

Double rotor &
\tikzmark{P10}$\Posset$ &
\tikzmark{E10}$\Eigensetabc$ &
\tikzmark{p10}$\Pivotsetone$ &
\sethighlight{$\Pivotsettwo$} &
- &
- &
\sethighlight{$\conesmax$} &
\sethighlight{$\conesmaxtwo$} &
3 (15) \\

\bottomrule


% Arrows.
\begin{tikzpicture}[overlay,remember picture]
     \draw[->,ForestGreen] (P1.center) to [out=-160,in=135](P2) to [out=-135,in=135](P3) to [out=-135,in=135](P4) to [out=-135,in=135](P5) to [out=-135,in=135](P6) to [out=-115,in=115](P10);
     \draw[->,ForestGreen] (P7.center) to [out=-20,in=45](P8) to [out=-45,in=45](P9);
     \draw[->,ForestGreen] (E2.center) to [out=-160,in=135](E3) to [out=-135,in=135](E7);
     \draw[->,ForestGreen] (E3b.center) to [out=0,in=45](E5) to [out=-45,in=45](E8);
     \draw[->,ForestGreen] (E4.center) to [out=-160,in=135](E6) to [out=-135,in=135](E9) to [out=-135,in=135](E10);
     \draw[->,ForestGreen] (p2.center) to [out=-160,in=135](p3) to [out=-135,in=135](p4) to [out=-135,in=135](p5) to [out=-135,in=135](p6) to [out=-135,in=135](p7) to [out=-135,in=135](p8) to [out=-135,in=135](p9) to [out=-135,in=135](p10);
     \draw[->,ForestGreen] (S3.center) to [out=-160,in=135](S4) to [out=-135,in=135](S5) to [out=-135,in=135](S8);
     \draw[->,ForestGreen] (S4b.center) to [out=-20,in=45](S6) to [out=-45,in=45](S9);
     \draw[->,ForestGreen] (Sy4.center) to [out=-20,in=45](Sy6) to [out=-45,in=45](Sy9);
     \draw[->,ForestGreen] (s2.center) to [out=-20,in=45](s3) to [out=-45,in=45](s4);
\end{tikzpicture}

\end{tabular}
\end{table}

\end{document}
