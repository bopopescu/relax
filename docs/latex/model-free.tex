% Model-free analysis.
%%%%%%%%%%%%%%%%%%%%%%

\chapter{Model-free analysis}
\index{model-free analysis|textbf}



% Theory.
%%%%%%%%%

\section{Theory}



% The chi-squared function.
\subsection{The chi-squared function -- $\chi^2(\theta)$}
\index{chi-squared|textbf}

For the minimisation\index{minimisation} of the model-free models a chain of calculations, each based on a different theory, is required.  At the highest level the function which is actually minimised is the chi-squared function
\begin{equation} \label{eq: chi-squared}
 \chi^2(\theta) = \sum_{i=1}^n \frac{(\Ri - \Ri(\theta))^2}{\sigma_i^2},
\end{equation}

\noindent where the index $i$ is the summation index ranging over all the experimentally collected relaxation data; $\Ri$ belongs to the set relaxation data set R for an individual residue, a collection of residues, or the entire macromolecule and includes the $\Rone$, $\Rtwo$, and NOE data at all field strengths; $\Ri(\theta)$ is the back-calculated relaxation value belonging to the set R$(\theta)$; $\theta$ is the model parameter vector which when minimised is denoted by $\hat\theta$; and $\sigma_i$ is the experimental error.

The significance of the chi-squared equation~\eqref{eq: chi-squared} is that the function returns a single value which is then minimised by the optimisation algorithm to find the model-free parameter values of the given model.



% The relaxation equations.
\subsection{The relaxation equations -- $\Ri(\theta)$}

The chi-squared equation is itself dependent on the relaxation equations through the back-calculated relaxation data R$(\theta)$.  These values include the spin-lattice\index{relaxation rate!spin-lattice}, spin-spin\index{relaxation rate!spin-spin}, and cross-relaxation rates\index{relaxation rate!cross rate} of \citet{Abragam61} and are respectively
\begin{align}
 \Rone(\theta) &= d \Big( J(\omega_H - \omega_X) + 3J(\omega_X) + 6J(\omega_H + \omega_X) \Big) + cJ(\omega_X),     \label{eq: R1} \\
 \Rtwo(\theta) &= \frac{d}{2} \Big( 4J(0) + J(\omega_H - \omega_X) + 3J(\omega_X) + 6J(\omega_H)                    \nonumber \\
     &  \quad + 6J(\omega_H + \omega_X) \Big) + \frac{c}{6} \Big( 4J(0) + 3J(\omega_X) \Big) + R_{ex},              \label{eq: R2} \\  
 \crossrate(\theta) &= d \Big( 6J(\omega_H + \omega_X) - J(\omega_H - \omega_X) \Big),                              \label{eq: sigma_NOE}
\end{align}

\noindent where $J(\omega)$ is the power spectral density function and $R_{ex}$ is the relaxation due to chemical exchange.  The dipolar and CSA constants are defined in SI units as
\begin{gather}
 d = \frac{1}{4} \left(\frac{\mu_0}{4\pi}\right)^2 \frac{(\gH \gX \hbar)^2}{\langle r^6 \rangle}, \label{eq: dipolar constant} \\
 c = \frac{(\omega_H \Delta\sigma)^2}{3}, \label{eq: CSA constant}
\end{gather}

\noindent where $\mu_0$ is the permeability of free space, $\gH$ and $\gX$ are the gyromagnetic ratios of the $H$ and $X$ spins respectively, $\hbar$ is Plank's constant divided by $2\pi$, $r$ is the bond length, and $\Delta\sigma$ is the chemical shift anisotropy measured in ppm.  The cross-relaxation rate $\crossrate$\index{relaxation rate!cross-relaxation|textbf} is related to the steady state NOE by the equation
\begin{equation} \label{eq: NOE}
 \mathrm{NOE}(\theta) = 1 + \frac{\gH}{\gX} \frac{\crossrate(\theta)}{\Rone(\theta)}.
\end{equation}



% The transformed relaxation equations.
\subsection{The transformed relaxation equations -- $\Ri'(\theta)$}

Letting the relaxation equations $\Ri(\theta)$ be the $\Rone(\theta)$, $\Rtwo(\theta)$, and NOE$(\theta)$ an additional layer of abstraction can be used to simply the calculation of the gradients and Hessians.  This involved decomposing the NOE equation into the cross relaxation rate constant $\crossrate(\theta)$ and the auto relaxation rate $\Rone(\theta)$.  Taking equation~\eqref{eq: NOE} the relaxation equations are
\begin{align}
 \Rone'(\theta) &= \Rone(\theta) \\
 \Rtwo'(\theta) &= \Rtwo(\theta) \\
 \mathrm{NOE}(\theta)  &= 1 + \frac{\gH}{\gX} \frac{\crossrate(\theta)}{\Rone(\theta)}.
\end{align}

\noindent whereas the transformed relaxation equations are \{$\Rone(\theta)$, $\Rtwo(\theta)$, $\crossrate(\theta)$\}.



% The spectral density functions.
\subsection{The spectral density functions -- $J(\omega)$}

The relaxation equations are themselves dependent on the calculation of the spectral density values $J(\omega)$.  Within model-free analysis these are modelled by the original model-free formula \citep{LipariSzabo82a, LipariSzabo82b}
\begin{equation} \label{eq: J(w) model-free generic}
 J(\omega) = \frac{2}{5} \sum_{i=-k}^k c_i \cdot \tau_i \Bigg(
  \frac{S^2}{1 + (\omega \tau_i)^2}
  + \frac{(1 - S^2)(\tau_e + \tau_i)\tau_e}{(\tau_e + \tau_i)^2 + (\omega \tau_e \tau_i)^2}
 \Bigg),
\end{equation}

\noindent where $S^2$ is the square of the Lipari and Szabo generalised order parameter and $\tau_e$ is the effective correlation time.  The order parameter reflects the amplitude of the motion and the correlation time in an indication of the time scale of that motion.  The theory was extended by \citet{Clore90} by the modelling of two independent internal motions using the equation
\begin{multline} \label{eq: J(w) model-free ext generic}
 J(\omega) = \frac{2}{5} \sum_{i=-k}^k c_i \cdot \tau_i \Bigg(
  \frac{S^2}{1 + (\omega \tau_i)^2}
  + \frac{(1 - S^2_f)(\tau_f + \tau_i)\tau_f}{(\tau_f + \tau_i)^2 + (\omega \tau_f \tau_i)^2}       \\
  + \frac{(S^2_f - S^2)(\tau_s + \tau_i)\tau_s}{(\tau_s + \tau_i)^2 + (\omega \tau_s \tau_i)^2}
 \Bigg).
\end{multline}

\noindent where $S^2_f$ and $\tau_f$ are the amplitude and timescale of the faster of the two motions whereas $S^2_s$ and $\tau_s$ are those of the slower motion.  $S^2_f$ and $S^2_s$ are related by the formula $S^2 = S^2_f \cdot S^2_s$.



% Brownian rotational diffusion.
\subsection{Brownian rotational diffusion}

\index{diffusion!Brownian|textbf}
In equations~\eqref{eq: J(w) model-free generic} and~\eqref{eq: J(w) model-free ext generic} the generic Brownian diffusion NMR correlation function presented in \citet{dAuvergneGooley06b} has been used.  This function is
\begin{equation} \label{eq: C(tau) generic}
 C(\tau) = \frac{1}{5} \sum_{i=-k}^k c_i \cdot e^{-\tau/\tau_i},
\end{equation}

\noindent where the summation index $i$ ranges over the number of exponential terms within the correlation function.  This equation is generic in that it can describe the diffusion\index{diffusion} of an ellipsoid, a spheroid, or a sphere.



% Diffusion as an ellipsoid.
\subsubsection{Diffusion as an ellipsoid}
\index{diffusion!ellipsoid (asymmetric)|textbf}

For the ellipsoid defined by the parameter set \{$\Diff_{iso}$, $\Diff_a$, $\Diff_r$, $\alpha$, $\beta$, $\gamma$\} the variable $k$ is equal to two and therefore the index $i \in \{-2, -1, 0, 1, 2\}$.  The geometric parameters \{$\Diff_{iso}$, $\Diff_a$, $\Diff_r$\} are defined as
\begin{subequations}
\begin{align}
 & \Diff_{iso} = \tfrac{1}{3} (\Diff_x + \Diff_y + \Diff_z ),   \label{eq: Diso ellipsoid def} \\
 & \Diff_a = \Diff_z - \tfrac{1}{2}(\Diff_x + \Diff_y),         \label{eq: Da ellipsoid def} \\
 & \Diff_r = \frac{\Diff_y - \Diff_x}{2\Diff_a},                \label{eq: Dr ellipsoid def}
\end{align}
\end{subequations}

\noindent and are constrained by
\begin{subequations}
\begin{align}
 0 & < \Diff_{iso} < \infty,                                                    \label{eq: Diso lim} \\
 0 & \le \Diff_a < \frac{\Diff_{iso}}{\tfrac{1}{3} + \Diff_r} \le 3\Diff_{iso}, \label{eq: Da lim} \\
 0 & \le \Diff_r \le 1.                                                         \label{eq: Dr lim}
\end{align}
\end{subequations}

\noindent The orientational parameters \{$\alpha$, $\beta$, $\gamma$\} are the Euler angles using the z-y-z rotation notation.


The five weights $c_i$ are defined as
\begin{subequations}
\begin{align}
 c_{-2} &= \tfrac{1}{4}(d + e),     \label{eq: ellipsoid c-2} \\
 c_{-1} &= 3\delta_y^2\delta_z^2,   \label{eq: ellipsoid c-1} \\
 c_{0}  &= 3\delta_x^2\delta_z^2,   \label{eq: ellipsoid c0} \\
 c_{1}  &= 3\delta_x^2\delta_y^2,   \label{eq: ellipsoid c1} \\
 c_{2}  &= \tfrac{1}{4}(d - e),     \label{eq: ellipsoid c2}
\end{align}
\end{subequations}

\noindent where
\begin{align}
 d &= 3 \left( \delta_x^4 + \delta_y^4 + \delta_z^4 \right) - 1, \label{eq: ellipsoid d} \\
 e &= -\frac{1}{\mathfrak{R}} \bigg[ (1 + 3\Diff_r) \left(\delta_x^4 + 2\delta_y^2\delta_z^2\right)
   + (1 - 3\Diff_r) \left(\delta_y^4 + 2\delta_x^2\delta_z^2\right) - 2 \left(\delta_z^4 + 2\delta_x^2\delta_y^2\right) \bigg], \label{eq: ellipsoid e}
\end{align}

\noindent and where
\begin{equation}
 \mathfrak{R} = \sqrt{1 + 3\Diff_r^2}.
\end{equation}


The five correlation times $\tau_i$ are
\begin{subequations}
\begin{align}
 1/\tau_{-2} &= 6 \Diff_{iso} - 2\Diff_a\mathfrak{R},   \label{eq: ellipsoid tau-2} \\
 1/\tau_{-1} &= 6 \Diff_{iso} - \Diff_a (1 + 3\Diff_r), \label{eq: ellipsoid tau-1} \\
 1/\tau_{0}  &= 6 \Diff_{iso} - \Diff_a (1 - 3\Diff_r), \label{eq: ellipsoid tau0} \\
 1/\tau_{1}  &= 6 \Diff_{iso} + 2\Diff_a,               \label{eq: ellipsoid tau1} \\
 1/\tau_{2}  &= 6 \Diff_{iso} + 2\Diff_a\mathfrak{R}.   \label{eq: ellipsoid tau2}
\end{align}
\end{subequations}



% Diffusion as a spheroid.
\subsubsection{Diffusion as a spheroid}
\index{diffusion!spheroid (axially symmetric)|textbf}

The variable $k$ is equal to one in the case of the spheroid\index{diffusion!spheroid (axially symmetric)|textbf} defined by the parameter set \{$\Diff_{iso}$, $\Diff_a$, $\theta$, $\phi$\}, hence $i \in \{-1, 0, 1\}$.  The geometric parameters \{$\Diff_{iso}$, $\Diff_a$\} are defined as
\begin{subequations}
\begin{align}
 & \Diff_{iso} = \tfrac{1}{3} (\Diff_\Par + 2\Diff_\Per),   \label{eq: Diso spheroid def} \\
 & \Diff_a = \Diff_\Par - \Diff_\Per.                       \label{eq: Da spheroid def}
\end{align}
\end{subequations}

\noindent and are constrained by
\begin{subequations}
\begin{gather}
 0 < \Diff_{iso} < \infty, \\
 -\tfrac{3}{2} \Diff_{iso} < \Diff_a < 3\Diff_{iso}.
\end{gather}
\end{subequations}

\noindent The orientational parameters \{$\theta$, $\phi$\} are the spherical angles defining the orientation of the major axis of the diffusion frame within the lab frame.


The three weights $c_i$ are
\begin{subequations}
\begin{align}
 c_{-1} &= \tfrac{1}{4}(3\delta_z^2 - 1)^2, \label{eq: spheroid c-1} \\
 c_{0}  &= 3\delta_z^2(1 - \delta_z^2),     \label{eq: spheroid c0} \\
 c_{1}  &= \tfrac{3}{4}(\delta_z^2 - 1)^2.  \label{eq: spheroid c1}
\end{align}
\end{subequations}

The five correlation times $\tau_i$ are
\begin{subequations}
\begin{align}
 1/\tau_{-1} &= 6\Diff_{iso} - 2\Diff_a,    \label{eq: spheroid tau-1} \\
 1/\tau_{0}  &= 6\Diff_{iso} - \Diff_a,     \label{eq: spheroid tau0} \\
 1/\tau_{1}  &= 6\Diff_{iso} + 2\Diff_a.    \label{eq: spheroid tau1}
\end{align}
\end{subequations}



% Diffusion as a sphere.
\subsubsection{Diffusion as a sphere}
\index{diffusion!sphere (isotropic)|textbf}

In the situation of a molecule diffusing as a sphere\index{diffusion!sphere (isotropic)|textbf} either described by the single parameter $\tau_m$ or $\Diff_{iso}$, the variable $k$ is equal to zero.  Therefore $i \in \{0\}$.  The single weight $c_0$ is equal to one and the single correlation time $\tau_0$ is equivalent to the global tumbling time $\tau_m$ given by
\begin{equation} \label{eq: sphere tau0}
 1/\tau_m = 6\Diff_{iso}.
\end{equation}

\noindent This is diffusion equation presented in \citet{Bloembergen48}.



% Optimisation of a single model-free model.
%%%%%%%%%%%%%%%%%%%%%%%%%%%%%%%%%%%%%%%%%%%%

\newpage
\section{Optimisation of a single model-free model}


% The sample script.
\subsection{The sample script}

The sample script which demonstrates the optimisation of model-free model $m4$ which consists of the parameters \{$S^2$, $\tau_e$, $R_{ex}$\} is \texttt{`model-free.py'}.  The text of the script is:

\begin{exampleenv}
\# Script for model-free analysis. \\
 \\
\# Create the run. \\
name = `m4' \\
run.create(name, `mf') \\
 \\
\# Nuclei type \\
nuclei(`N') \\
 \\
\# Load the sequence. \\
sequence.read(name, `noe.500.out') \\
 \\
\# Load the relaxation data. \\
relax\_data.read(name, `R1', `600', 600.0 * 1e6, `r1.600.out') \\
relax\_data.read(name, `R2', `600', 600.0 * 1e6, `r2.600.out') \\
relax\_data.read(name, `NOE', `600', 600.0 * 1e6, `noe.600.out') \\
relax\_data.read(name, `R1', `500', 500.0 * 1e6, `r1.500.out') \\
relax\_data.read(name, `R2', `500', 500.0 * 1e6, `r2.500.out') \\
relax\_data.read(name, `NOE', `500', 500.0 * 1e6, `noe.500.out') \\
 \\
\# Setup other values. \\
diffusion\_tensor.init(name, 10e-9, fixed=1) \\
value.set(name, 1.02 * 1e-10, `bond\_length') \\
value.set(name, -160 * 1e-6, `csa') \\
 \\
\# Select the model-free model. \\
model\_free.select\_model(run=name, model=name) \\
 \\
\# Grid search. \\
grid\_search(name, inc=11) \\
 \\
\# Minimise. \\
minimise(`newton', run=name) \\
 \\
\# Monte Carlo simulations. \\
monte\_carlo.setup(name, number=100) \\
monte\_carlo.create\_data(name) \\
monte\_carlo.initial\_values(name) \\
minimise(`newton', run=name) \\
eliminate(run=name) \\
monte\_carlo.error\_analysis(name) \\
 \\
\# Finish. \\
results.write(run=name, file=`results', force=1) \\
state.save(`save', force=1)
\end{exampleenv}


% The rest.
\subsection{The rest}

\textbf{\textit{Please write me!}}

Until this section is completed please look at the sample script \texttt{`model-free.py'}.



% Optimisation of all model-free models.
%%%%%%%%%%%%%%%%%%%%%%%%%%%%%%%%%%%%%%%%

\newpage
\section{Optimisation of all model-free models}


% The sample script.
\subsection{The sample script}

The sample script which demonstrates the optimisation of all model-free models from $m0$ to $m9$ of individual residues is \texttt{`mf\_multimodel.py'}.  The text of the script is:

\begin{exampleenv}
\# Script for model-free analysis. \\
 \\
\# Set the run names (also the names of preset model-free models). \\
runs = [`m0', `m1', `m2', `m3', `m4', `m5', `m6', `m7', `m8', `m9'] \\
 \\
\# Nuclei type \\
nuclei(`N') \\
 \\
\# Loop over the runs. \\
for name in runs: \\
\hspace*{4ex} \# Create the run. \\
\hspace*{4ex} run.create(name, `mf') \\
 \\
\hspace*{4ex} \# Load the sequence. \\
\hspace*{4ex} sequence.read(name, `noe.500.out') \\
 \\
\hspace*{4ex} \# Load the relaxation data. \\
\hspace*{4ex} relax\_data.read(name, `R1', `600', 600.0 * 1e6, `r1.600.out') \\
\hspace*{4ex} relax\_data.read(name, `R2', `600', 600.0 * 1e6, `r2.600.out') \\
\hspace*{4ex} relax\_data.read(name, `NOE', `600', 600.0 * 1e6, `noe.600.out') \\
\hspace*{4ex} relax\_data.read(name, `R1', `500', 500.0 * 1e6, `r1.500.out') \\
\hspace*{4ex} relax\_data.read(name, `R2', `500', 500.0 * 1e6, `r2.500.out') \\
\hspace*{4ex} relax\_data.read(name, `NOE', `500', 500.0 * 1e6, `noe.500.out') \\
 \\
\hspace*{4ex} \# Setup other values. \\
\hspace*{4ex} diffusion\_tensor.init(name, 1e-8, fixed=1) \\
\hspace*{4ex} value.set(name, 1.02 * 1e-10, `bond\_length') \\
\hspace*{4ex} value.set(name, -160 * 1e-6, `csa') \\
 \\
\hspace*{4ex} \# Select the model-free model. \\
\hspace*{4ex} model\_free.select\_model(run=name, model=name) \\
 \\
\hspace*{4ex} \# Minimise. \\
\hspace*{4ex} grid\_search(name, inc=11) \\
\hspace*{4ex} minimise(`newton', run=name) \\
 \\
\hspace*{4ex} \# Write the results. \\
\hspace*{4ex} results.write(run=name, file=`results', force=1) \\
 \\
\# Save the program state. \\
state.save(`save', force=1)
\end{exampleenv}


% The rest.
\subsection{The rest}

\textbf{\textit{Please write me!}}

Until this section is completed please look at the sample script \texttt{`mf\_multimodel.py'}.



% Model-free model selection.
%%%%%%%%%%%%%%%%%%%%%%%%%%%%%

\newpage
\section{Model-free model selection}


% The sample script.
\subsection{The sample script}

The sample script which demonstrates both model-free model elimination and model-free model selection between models from $m0$ to $m9$ is \texttt{`modsel.py'}.  The text of the script is:

\begin{exampleenv}
\# Script for model-free model selection. \\
 \\
\# Nuclei type \\
nuclei(`N') \\
 \\
\# Set the run names. \\
runs = [`m0', `m1', `m2', `m3', `m4', `m5', `m6', `m7', `m8', `m9'] \\
 \\
\# Loop over the run names. \\
for name in runs: \\
\hspace*{4ex} print ``$\backslash$n$\backslash$n\# '' + name + `` \#'' \\
 \\
\hspace*{4ex} \# Create the run. \\
\hspace*{4ex} run.create(name, `mf') \\
 \\
\hspace*{4ex} \# Reload precalculated results from the file `m1/results', etc. \\
\hspace*{4ex} results.read(run=name, file=`results', dir=name) \\
 \\
\# Model elimination. \\
eliminate() \\
 \\
\# Model selection. \\
run.create(`aic', `mf') \\
model\_selection(`AIC', `aic') \\
 \\
\# Write the results. \\
state.save(`save', force=1) \\
results.write(run=`aic', file=`results', force=1)
\end{exampleenv}


% The rest.
\subsection{The rest}

\textbf{\textit{Please write me!}}

Until this section is completed please look at the sample script \texttt{`modsel.py'}.



% The new model-free optimisations protocol.
%%%%%%%%%%%%%%%%%%%%%%%%%%%%%%%%%%%%%%%%%%%%

\newpage
\section{The new model-free optimisation protocol}

\textbf{\textit{Please write me!}}

Until this section is written please look at the sample script \texttt{`full\_analysis.py'}.
