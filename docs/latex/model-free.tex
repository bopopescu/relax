% Model-free analysis.
%%%%%%%%%%%%%%%%%%%%%%

\chapter{Model-free analysis}
\index{model-free analysis|textbf}



% Theory.
%%%%%%%%%

\section{Theory}



% The chi-squared function.
\subsection{The chi-squared function -- $\chi^2(\theta)$}
\index{chi-squared|textbf}

For the minimisation\index{minimisation} of the model-free models a chain of calculations, each based on a different theory, is required.  At the highest level the equation which is actually minimised is the chi-squared function
\begin{equation} \label{eq: chi-squared}
 \chi^2(\theta) = \sum_{i=1}^n \frac{(\Ri - \Ri(\theta))^2}{\sigma_i^2},
\end{equation}

\noindent where the index $i$ is the summation index ranging over all the experimentally collected relaxation data of all residues used in the analysis; $\Ri$ belongs to the relaxation data set R for an individual residue, a collection of residues, or the entire macromolecule and includes the $\Rone$, $\Rtwo$, and NOE data at all field strengths; $\Ri(\theta)$ is the back-calculated relaxation value belonging to the set R$(\theta)$; $\theta$ is the model parameter vector which when minimised is denoted by $\hat\theta$; and $\sigma_i$ is the experimental error.

The significance of the chi-squared equation~\eqref{eq: chi-squared} is that the function returns a single value which is then minimised by the optimisation algorithm to find the model-free parameter values of the given model.



% The transformed relaxation equations.
\subsection{The transformed relaxation equations -- $\Ri(\theta)$}

The chi-squared equation is itself dependent on the relaxation equations through the back-calculated relaxation data R$(\theta)$.  Letting the relaxation values of the set R$(\theta)$ be the $\Rone(\theta)$, $\Rtwo(\theta)$, and NOE$(\theta)$ an additional layer of abstraction can be used to simplify the calculation of the gradients and Hessians.  This involves decomposing the NOE equation into the cross relaxation rate constant $\crossrate(\theta)$ and the auto relaxation rate $\Rone(\theta)$.  Taking equation~\eqref{eq: NOE} below the transformed relaxation equations are
\begin{subequations}
\begin{align}
    \Rone(\theta) &= \Rone'(\theta), \\
    \Rtwo(\theta) &= \Rtwo'(\theta), \\
    \mathrm{NOE}(\theta)  &= 1 + \frac{\gH}{\gX} \frac{\crossrate(\theta)}{\Rone(\theta)}.
\end{align}
\end{subequations}

\noindent whereas the relaxation equations are the $\Rone(\theta)$, $\Rtwo(\theta)$, $\crossrate(\theta)$.



% The relaxation equations.
\subsection{The relaxation equations -- $\Ri'(\theta)$}

The relaxation values of the set R$'(\theta)$ include the spin-lattice\index{relaxation rate!spin-lattice}, spin-spin\index{relaxation rate!spin-spin}, and cross-relaxation rates\index{relaxation rate!cross rate} at all field strengths.  These rates are respectively \citep{Abragam61}
\begin{subequations}
\begin{align}
    \Rone(\theta) &= d \Big( J(\omega_H - \omega_X) + 3J(\omega_X) + 6J(\omega_H + \omega_X) \Big) + cJ(\omega_X),     \label{eq: R1} \\
    \Rtwo(\theta) &= \frac{d}{2} \Big( 4J(0) + J(\omega_H - \omega_X) + 3J(\omega_X) + 6J(\omega_H)                    \nonumber \\
        & \quad + 6J(\omega_H + \omega_X) \Big) + \frac{c}{6} \Big( 4J(0) + 3J(\omega_X) \Big) + R_{ex},              \label{eq: R2} \\  
    \crossrate(\theta) &= d \Big( 6J(\omega_H + \omega_X) - J(\omega_H - \omega_X) \Big),                              \label{eq: sigma_NOE}
\end{align}
\end{subequations}

\noindent where $J(\omega)$ is the power spectral density function and $R_{ex}$ is the relaxation due to chemical exchange.  The dipolar and CSA constants are defined in SI units as
\begin{gather}
 d = \frac{1}{4} \left(\frac{\mu_0}{4\pi}\right)^2 \frac{(\gH \gX \hbar)^2}{\langle r^6 \rangle}, \label{eq: dipolar constant} \\
 c = \frac{(\omega_H \Delta\sigma)^2}{3}, \label{eq: CSA constant}
\end{gather}

\noindent where $\mu_0$ is the permeability of free space, $\gH$ and $\gX$ are the gyromagnetic ratios of the $H$ and $X$ spins respectively, $\hbar$ is Plank's constant divided by $2\pi$, $r$ is the bond length, and $\Delta\sigma$ is the chemical shift anisotropy measured in ppm.  The cross-relaxation rate $\crossrate$\index{relaxation rate!cross-relaxation|textbf} is related to the steady state NOE by the equation
\begin{equation} \label{eq: NOE}
 \mathrm{NOE}(\theta) = 1 + \frac{\gH}{\gX} \frac{\crossrate(\theta)}{\Rone(\theta)}.
\end{equation}



% The spectral density functions.
\subsection{The spectral density functions -- $J(\omega)$}

The relaxation equations are themselves dependent on the calculation of the spectral density values $J(\omega)$.  Within model-free analysis these are modelled by the original model-free formula \citep{LipariSzabo82a, LipariSzabo82b}
\begin{equation} \label{eq: J(w) model-free generic}
    J(\omega) = \frac{2}{5} \sum_{i=-k}^k c_i \cdot \tau_i \Bigg(
        \frac{S^2}{1 + (\omega \tau_i)^2}
        + \frac{(1 - S^2)(\tau_e + \tau_i)\tau_e}{(\tau_e + \tau_i)^2 + (\omega \tau_e \tau_i)^2}
    \Bigg),
\end{equation}

\noindent where $S^2$ is the square of the Lipari and Szabo generalised order parameter and $\tau_e$ is the effective correlation time.  The order parameter reflects the amplitude of the motion and the correlation time in an indication of the time scale of that motion.  The theory was extended by \citet{Clore90a} by the modelling of two independent internal motions using the equation
\begin{multline} \label{eq: J(w) model-free ext generic}
    J(\omega) = \frac{2}{5} \sum_{i=-k}^k c_i \cdot \tau_i \Bigg(
        \frac{S^2}{1 + (\omega \tau_i)^2}
        + \frac{(1 - S^2_f)(\tau_f + \tau_i)\tau_f}{(\tau_f + \tau_i)^2 + (\omega \tau_f \tau_i)^2}       \\
        + \frac{(S^2_f - S^2)(\tau_s + \tau_i)\tau_s}{(\tau_s + \tau_i)^2 + (\omega \tau_s \tau_i)^2}
    \Bigg).
\end{multline}

\noindent where $S^2_f$ and $\tau_f$ are the amplitude and timescale of the faster of the two motions whereas $S^2_s$ and $\tau_s$ are those of the slower motion.  $S^2_f$ and $S^2_s$ are related by the formula $S^2 = S^2_f \cdot S^2_s$.



% Brownian rotational diffusion.
\subsection{Brownian rotational diffusion}

\index{diffusion!Brownian|textbf}
In equations~\eqref{eq: J(w) model-free generic} and~\eqref{eq: J(w) model-free ext generic} the generic Brownian diffusion NMR correlation function presented in \citet{dAuvergneGooley06b} has been used.  This function is
\begin{equation} \label{eq: C(tau) generic}
    C(\tau) = \frac{1}{5} \sum_{i=-k}^k c_i \cdot e^{-\tau/\tau_i},
\end{equation}

\noindent where the summation index $i$ ranges over the number of exponential terms within the correlation function.  This equation is generic in that it can describe the diffusion\index{diffusion} of an ellipsoid, a spheroid, or a sphere.



% Diffusion as an ellipsoid.
\subsubsection{Diffusion as an ellipsoid}
\index{diffusion!ellipsoid (asymmetric)|textbf}

For the ellipsoid defined by the parameter set \{$\Diff_{iso}$, $\Diff_a$, $\Diff_r$, $\alpha$, $\beta$, $\gamma$\} the variable $k$ is equal to two and therefore the index $i \in \{-2, -1, 0, 1, 2\}$.  The geometric parameters \{$\Diff_{iso}$, $\Diff_a$, $\Diff_r$\} are defined as
\begin{subequations}
\begin{align}
    & \Diff_{iso} = \tfrac{1}{3} (\Diff_x + \Diff_y + \Diff_z ),   \label{eq: Diso ellipsoid def} \\
    & \Diff_a = \Diff_z - \tfrac{1}{2}(\Diff_x + \Diff_y),         \label{eq: Da ellipsoid def} \\
    & \Diff_r = \frac{\Diff_y - \Diff_x}{2\Diff_a},                \label{eq: Dr ellipsoid def}
\end{align}
\end{subequations}

\noindent and are constrained by
\begin{subequations}
\begin{align}
    0 & < \Diff_{iso} < \infty,                                                    \label{eq: Diso lim} \\
    0 & \le \Diff_a < \frac{\Diff_{iso}}{\tfrac{1}{3} + \Diff_r} \le 3\Diff_{iso}, \label{eq: Da lim} \\
    0 & \le \Diff_r \le 1.                                                         \label{eq: Dr lim}
\end{align}
\end{subequations}

\noindent The orientational parameters \{$\alpha$, $\beta$, $\gamma$\} are the Euler angles using the z-y-z rotation notation.


The five weights $c_i$ are defined as
\begin{subequations}
\begin{align}
    c_{-2} &= \tfrac{1}{4}(d - e),     \label{eq: ellipsoid c-2} \\
    c_{-1} &= 3\delta_y^2\delta_z^2,   \label{eq: ellipsoid c-1} \\
    c_{0}  &= 3\delta_x^2\delta_z^2,   \label{eq: ellipsoid c0} \\
    c_{1}  &= 3\delta_x^2\delta_y^2,   \label{eq: ellipsoid c1} \\
    c_{2}  &= \tfrac{1}{4}(d + e),     \label{eq: ellipsoid c2}
\end{align}
\end{subequations}

\noindent where
\begin{align}
    d &= 3 \left( \delta_x^4 + \delta_y^4 + \delta_z^4 \right) - 1, \label{eq: ellipsoid d} \\
    e &= \frac{1}{\mathfrak{R}} \bigg[ (1 + 3\Diff_r) \left(\delta_x^4 + 2\delta_y^2\delta_z^2\right)
        + (1 - 3\Diff_r) \left(\delta_y^4 + 2\delta_x^2\delta_z^2\right) - 2 \left(\delta_z^4 + 2\delta_x^2\delta_y^2\right) \bigg], \label{eq: ellipsoid e}
\end{align}

\noindent and where
\begin{equation}
    \mathfrak{R} = \sqrt{1 + 3\Diff_r^2}.
\end{equation}


The five correlation times $\tau_i$ are
\begin{subequations}
\begin{align}
    1/\tau_{-2} &= 6 \Diff_{iso} - 2\Diff_a\mathfrak{R},   \label{eq: ellipsoid tau-2} \\
    1/\tau_{-1} &= 6 \Diff_{iso} - \Diff_a (1 + 3\Diff_r), \label{eq: ellipsoid tau-1} \\
    1/\tau_{0}  &= 6 \Diff_{iso} - \Diff_a (1 - 3\Diff_r), \label{eq: ellipsoid tau0} \\
    1/\tau_{1}  &= 6 \Diff_{iso} + 2\Diff_a,               \label{eq: ellipsoid tau1} \\
    1/\tau_{2}  &= 6 \Diff_{iso} + 2\Diff_a\mathfrak{R}.   \label{eq: ellipsoid tau2}
\end{align}
\end{subequations}



% Diffusion as a spheroid.
\subsubsection{Diffusion as a spheroid}
\index{diffusion!spheroid (axially symmetric)|textbf}

The variable $k$ is equal to one in the case of the spheroid\index{diffusion!spheroid (axially symmetric)|textbf} defined by the parameter set \{$\Diff_{iso}$, $\Diff_a$, $\theta$, $\phi$\}, hence $i \in \{-1, 0, 1\}$.  The geometric parameters \{$\Diff_{iso}$, $\Diff_a$\} are defined as
\begin{subequations}
\begin{align}
    & \Diff_{iso} = \tfrac{1}{3} (\Diff_\Par + 2\Diff_\Per),   \label{eq: Diso spheroid def} \\
    & \Diff_a = \Diff_\Par - \Diff_\Per.                       \label{eq: Da spheroid def}
\end{align}
\end{subequations}

\noindent and are constrained by
\begin{subequations}
\begin{gather}
    0 < \Diff_{iso} < \infty, \\
    -\tfrac{3}{2} \Diff_{iso} < \Diff_a < 3\Diff_{iso}.
\end{gather}
\end{subequations}

\noindent The orientational parameters \{$\theta$, $\phi$\} are the spherical angles defining the orientation of the major axis of the diffusion frame within the lab frame.


The three weights $c_i$ are
\begin{subequations}
\begin{align}
    c_{-1} &= \tfrac{1}{4}(3\delta_z^2 - 1)^2, \label{eq: spheroid c-1} \\
    c_{0}  &= 3\delta_z^2(1 - \delta_z^2),     \label{eq: spheroid c0} \\
    c_{1}  &= \tfrac{3}{4}(\delta_z^2 - 1)^2.  \label{eq: spheroid c1}
\end{align}
\end{subequations}

The five correlation times $\tau_i$ are
\begin{subequations}
\begin{align}
    1/\tau_{-1} &= 6\Diff_{iso} - 2\Diff_a,    \label{eq: spheroid tau-1} \\
    1/\tau_{0}  &= 6\Diff_{iso} - \Diff_a,     \label{eq: spheroid tau0} \\
    1/\tau_{1}  &= 6\Diff_{iso} + 2\Diff_a.    \label{eq: spheroid tau1}
\end{align}
\end{subequations}



% Diffusion as a sphere.
\subsubsection{Diffusion as a sphere}
\index{diffusion!sphere (isotropic)|textbf}

In the situation of a molecule diffusing as a sphere\index{diffusion!sphere (isotropic)|textbf} either described by the single parameter $\tau_m$ or $\Diff_{iso}$, the variable $k$ is equal to zero.  Therefore $i \in \{0\}$.  The single weight $c_0$ is equal to one and the single correlation time $\tau_0$ is equivalent to the global tumbling time $\tau_m$ given by
\begin{equation} \label{eq: sphere tau0}
    1/\tau_m = 6\Diff_{iso}.
\end{equation}

\noindent This is diffusion equation presented in \citet{Bloembergen48}.


% The model-free models.
%~~~~~~~~~~~~~~~~~~~~~~~

\subsection{The model-free models}

Extending the list of models given in \citet{Mandel95, Fushman97, Orekhov99b, Korzhnev01, Zhuravleva04}, the models built into relax include
\begin{subequations}
\renewcommand{\theequation}{\theparentequation .\arabic{equation}}
\addtocounter{equation}{-1}
\begin{align}
 m0 &= \{\},                                   \label{model: m0} \\
 m1 &= \{S^2\},                                \label{model: m1} \\
 m2 &= \{S^2, \tau_e\},                        \label{model: m2} \\
 m3 &= \{S^2, R_{ex}\},                        \label{model: m3} \\
 m4 &= \{S^2, \tau_e, R_{ex}\},                \label{model: m4} \\
 m5 &= \{S^2, S^2_f, \tau_s\},                 \label{model: m5} \\
 m6 &= \{S^2, \tau_f, S^2_f, \tau_s\},         \label{model: m6} \\
 m7 &= \{S^2, S^2_f, \tau_s, R_{ex}\},         \label{model: m7} \\
 m8 &= \{S^2, \tau_f, S^2_f, \tau_s, R_{ex}\}, \label{model: m8} \\
 m9 &= \{R_{ex}\}.                             \label{model: m9}
\end{align}
\end{subequations}

\noindent The parameter $R_{ex}$ is scaled quadratically with field strength in these models as it is assumed to be fast.  In the set theory notation, the model-free model for the spin system $i$ is represented by the symbol $\Mfset_i$.  Through the addition of the local $\tau_m$ to each of these models, only the component of Brownian rotational diffusion experienced by the spin system is probed.  These models, represented in set notation by the symbol $\Localset_i$, are
\begin{subequations}
\renewcommand{\theequation}{\theparentequation .\arabic{equation}}
\addtocounter{equation}{-1}
\begin{align}
 tm0 &= \{\tau_m\},                                     \label{model: tm0} \\
 tm1 &= \{\tau_m, S^2\},                                \label{model: tm1} \\
 tm2 &= \{\tau_m, S^2, \tau_e\},                        \label{model: tm2} \\
 tm3 &= \{\tau_m, S^2, R_{ex}\},                        \label{model: tm3} \\
 tm4 &= \{\tau_m, S^2, \tau_e, R_{ex}\},                \label{model: tm4} \\
 tm5 &= \{\tau_m, S^2, S^2_f, \tau_s\},                 \label{model: tm5} \\
 tm6 &= \{\tau_m, S^2, \tau_f, S^2_f, \tau_s\},         \label{model: tm6} \\
 tm7 &= \{\tau_m, S^2, S^2_f, \tau_s, R_{ex}\},         \label{model: tm7} \\
 tm8 &= \{\tau_m, S^2, \tau_f, S^2_f, \tau_s, R_{ex}\}, \label{model: tm8} \\
 tm9 &= \{\tau_m, R_{ex}\}.                             \label{model: tm9}
\end{align}
\end{subequations}




% Optimisation of a single model-free model.
%%%%%%%%%%%%%%%%%%%%%%%%%%%%%%%%%%%%%%%%%%%%

\section{Optimisation of a single model-free model}


% The sample script.
%~~~~~~~~~~~~~~~~~~~

\subsection{The sample script}

The sample script which demonstrates the optimisation of model-free model $m4$ which consists of the parameters \{$S^2$, $\tau_e$, $R_{ex}$\} is \texttt{`model-free.py'}.  The text of the script is:

\begin{exampleenv}
\# Script for model-free analysis. \\
 \\
\# Create the run. \\
name = `m4' \\
run.create(name, `mf') \\
 \\
\# Nuclei type \\
nuclei(`N') \\
 \\
\# Load the sequence. \\
sequence.read(name, `noe.500.out') \\
 \\
\# Load the relaxation data. \\
relax\_data.read(name, `R1', `600', 600.0 * 1e6, `r1.600.out') \\
relax\_data.read(name, `R2', `600', 600.0 * 1e6, `r2.600.out') \\
relax\_data.read(name, `NOE', `600', 600.0 * 1e6, `noe.600.out') \\
relax\_data.read(name, `R1', `500', 500.0 * 1e6, `r1.500.out') \\
relax\_data.read(name, `R2', `500', 500.0 * 1e6, `r2.500.out') \\
relax\_data.read(name, `NOE', `500', 500.0 * 1e6, `noe.500.out') \\
 \\
\# Setup other values. \\
diffusion\_tensor.init(name, 10e-9, fixed=1) \\
value.set(name, 1.02 * 1e-10, `bond\_length') \\
value.set(name, -160 * 1e-6, `csa') \\
 \\
\# Select the model-free model. \\
model\_free.select\_model(run=name, model=name) \\
 \\
\# Grid search. \\
grid\_search(name, inc=11) \\
 \\
\# Minimise. \\
minimise(`newton', run=name) \\
 \\
\# Monte Carlo simulations. \\
monte\_carlo.setup(name, number=100) \\
monte\_carlo.create\_data(name) \\
monte\_carlo.initial\_values(name) \\
minimise(`newton', run=name) \\
eliminate(run=name) \\
monte\_carlo.error\_analysis(name) \\
 \\
\# Finish. \\
results.write(run=name, file=`results', force=1) \\
state.save(`save', force=1)
\end{exampleenv}


% The rest.
%~~~~~~~~~~

\subsection{The rest}

\textbf{\textit{Please write me!}}

Until this section is completed please look at the sample script \texttt{`model-free.py'}.



% Optimisation of all model-free models.
%%%%%%%%%%%%%%%%%%%%%%%%%%%%%%%%%%%%%%%%

\section{Optimisation of all model-free models}


% The sample script.
%~~~~~~~~~~~~~~~~~~~

\subsection{The sample script}

The sample script which demonstrates the optimisation of all model-free models from $m0$ to $m9$ of individual residues is \texttt{`mf\_multimodel.py'}.  The text of the script is:

\begin{exampleenv}
\# Script for model-free analysis. \\
 \\
\# Set the run names (also the names of preset model-free models). \\
runs = [`m0', `m1', `m2', `m3', `m4', `m5', `m6', `m7', `m8', `m9'] \\
 \\
\# Nuclei type \\
nuclei(`N') \\
 \\
\# Loop over the runs. \\
for name in runs: \\
\hspace*{4ex} \# Create the run. \\
\hspace*{4ex} run.create(name, `mf') \\
 \\
\hspace*{4ex} \# Load the sequence. \\
\hspace*{4ex} sequence.read(name, `noe.500.out') \\
 \\
\hspace*{4ex} \# Load the relaxation data. \\
\hspace*{4ex} relax\_data.read(name, `R1', `600', 600.0 * 1e6, `r1.600.out') \\
\hspace*{4ex} relax\_data.read(name, `R2', `600', 600.0 * 1e6, `r2.600.out') \\
\hspace*{4ex} relax\_data.read(name, `NOE', `600', 600.0 * 1e6, `noe.600.out') \\
\hspace*{4ex} relax\_data.read(name, `R1', `500', 500.0 * 1e6, `r1.500.out') \\
\hspace*{4ex} relax\_data.read(name, `R2', `500', 500.0 * 1e6, `r2.500.out') \\
\hspace*{4ex} relax\_data.read(name, `NOE', `500', 500.0 * 1e6, `noe.500.out') \\
 \\
\hspace*{4ex} \# Setup other values. \\
\hspace*{4ex} diffusion\_tensor.init(name, 1e-8, fixed=1) \\
\hspace*{4ex} value.set(name, 1.02 * 1e-10, `bond\_length') \\
\hspace*{4ex} value.set(name, -160 * 1e-6, `csa') \\
 \\
\hspace*{4ex} \# Select the model-free model. \\
\hspace*{4ex} model\_free.select\_model(run=name, model=name) \\
 \\
\hspace*{4ex} \# Minimise. \\
\hspace*{4ex} grid\_search(name, inc=11) \\
\hspace*{4ex} minimise(`newton', run=name) \\
 \\
\hspace*{4ex} \# Write the results. \\
\hspace*{4ex} results.write(run=name, file=`results', force=1) \\
 \\
\# Save the program state. \\
state.save(`save', force=1)
\end{exampleenv}


% The rest.
%~~~~~~~~~~

\subsection{The rest}

\textbf{\textit{Please write me!}}

Until this section is completed please look at the sample script \texttt{`mf\_multimodel.py'}.



% Model-free model selection.
%%%%%%%%%%%%%%%%%%%%%%%%%%%%%

\section{Model-free model selection}


% The sample script.
%~~~~~~~~~~~~~~~~~~~

\subsection{The sample script}

The sample script which demonstrates both model-free model elimination and model-free model selection between models from $m0$ to $m9$ is \texttt{`modsel.py'}.  The text of the script is:

\begin{exampleenv}
\# Script for model-free model selection. \\
 \\
\# Nuclei type \\
nuclei(`N') \\
 \\
\# Set the run names. \\
runs = [`m0', `m1', `m2', `m3', `m4', `m5', `m6', `m7', `m8', `m9'] \\
 \\
\# Loop over the run names. \\
for name in runs: \\
\hspace*{4ex} print ``$\backslash$n$\backslash$n\# '' + name + `` \#'' \\
 \\
\hspace*{4ex} \# Create the run. \\
\hspace*{4ex} run.create(name, `mf') \\
 \\
\hspace*{4ex} \# Reload precalculated results from the file `m1/results', etc. \\
\hspace*{4ex} results.read(run=name, file=`results', dir=name) \\
 \\
\# Model elimination. \\
eliminate() \\
 \\
\# Model selection. \\
run.create(`aic', `mf') \\
model\_selection(`AIC', `aic') \\
 \\
\# Write the results. \\
state.save(`save', force=1) \\
results.write(run=`aic', file=`results', force=1)
\end{exampleenv}


% The rest.
%~~~~~~~~~~

\subsection{The rest}

\textbf{\textit{Please write me!}}

Until this section is completed please look at the sample script \texttt{`modsel.py'}.



% The methodology of Mandel et al., 1995.
%%%%%%%%%%%%%%%%%%%%%%%%%%%%%%%%%%%%%%%%%

\section{The methodology of Mandel et al., 1995}

By presenting a systematic methodology for obtaining a consistent model-free description of the dynamics of the system, the manuscript of \citet{Mandel95} revolutionised the application of model-free analysis.  The full protocol is presented in Figure~\ref{fig: Mandel et al.}.

All of the data analysis techniques required for this protocol can be implemented within relax.  The chi-squared distributions required for the chi-squared tests are constructed by Modelfree4 from the Monte Carlo simulations.  If the optimisation algorithms and Monte Carlo simulations built into relax are utilised, then the relax script will need to construct the chi-squared distributions from the results as this is not yet coded into relax.  The specific step-up hypothesis testing model selection of \citet{Mandel95} is available through the \texttt{model\_selection()} user function.  Coding the rest of the protocol into a script should be straightforward.



% Mandel et al., 1995 figure.
\begin{figure}
\centerline{\includegraphics[width=0.8\textwidth, bb=0 0 436 539]{images/model_free/mandel95.eps.gz}}
\caption[A schematic of the model-free optimisation protocol of Mandel et al., 1995]{A schematic of the model-free optimisation protocol of \citet{Mandel95}.  This specific protocol is for single field strength data.  The initial diffusion tensor estimate is calculated using the $\Rtwo/\Rone$ ratio.  The diffusion parameters of $\Diffset$ are held constant while model-free models $m1$ to $m5$ (\ref{model: m1}--\ref{model: m5}) of the set $\Mfset_i$ for each residue $i$ are optimised and 500 Monte Carlo simulations executed.  Using a web of ANOVA statistical tests, specifically $\chi^2$ and F-tests, a step-up hypothesis testing model selection procedure is used to choose the best model-free model.  These steps are repeated for all residues of the protein.  The global model $\Space$, the union of $\Diffset$ and all $\Mfset_i$, is then optimised.  These steps are repeated until convergence of the global model.  The iterative process is repeated for both isotropic diffusion (sphere) and anisotropic diffusion (spheroid).} \label{fig: Mandel et al.}
\end{figure}



% Kay's paradigm -- the initial diffusion tensor estimate.
%%%%%%%%%%%%%%%%%%%%%%%%%%%%%%%%%%%%%%%%%%%%%%%%%%%%%%%%%%

\section{Kay's paradigm -- the initial diffusion tensor estimate}

Ever since \citet{Kay89}, the question of how to obtain the model-free description of the system has followed the route in which the diffusion tensor is initially estimated.  Using this rough estimate, the model-free models are optimised for each spin system $i$, the best model selected, and then the global model $\Space$ of the diffusion model $\Diffset$ with each model-free model $\Mfset_i$ is optimised.  This procedure is then repeated using the diffusion tensor parameters of $\Space$ as the initial input.  Finally the global model is selected.  The full protocol is illustrated in Figure~\ref{fig: init diff estimate}.


% Kay's paradigm figure.
\begin{figure}
\centerline{\includegraphics[width=0.8\textwidth, bb=0 0 437 523]{images/model_free/init_diff_est.eps.gz}}
\caption[Model-free analysis using Kay's paradigm -- the initial diffusion tensor estimate]{A schematic of model-free analysis using Kay's paradigm -- the initial diffusion tensor estimate -- together with AIC model selection and model elimination.  The initial estimates of the parameters of $\Diffset$ are held constant while model-free models $m0$ to $m9$ (\ref{model: m0}--\ref{model: m9}) of the set $\Mfset_i$ for each spin system $i$ are optimised, model elimination applied to remove failed models, and AIC model selection used to determine the best model.  The global model $\Space$, the union of $\Diffset$ and all $\Mfset_i$, is then optimised.  These steps are repeated until convergence of the global model.  The entire iterative process is repeated for each of the Brownian diffusion models.  Finally AIC model selection is used to determine the best description of the dynamics of the molecule by selecting between the global models $\Space$ including the sphere, oblate spheroid, prolate spheroid, and ellipsoid.  Once the solution has been found, Monte Carlo simulations can be utilised for error analysis.} \label{fig: init diff estimate}
\end{figure}



% The new model-free optimisations protocol.
%%%%%%%%%%%%%%%%%%%%%%%%%%%%%%%%%%%%%%%%%%%%

\section{The new model-free optimisation protocol}

\textbf{\textit{Please write me!}}

Until this section is written please look at the sample script \texttt{`full\_analysis.py'}.  A description of the protocol is given at the top of the script.  The protocol is summarised in Figure~\ref{fig: new protocol}.


% New model-free optimisation protocol figure.
\begin{figure}
\centerline{\includegraphics[width=0.8\textwidth, bb=0 0 461 697]{images/model_free/new_protocol.eps.gz}}
\caption[A schematic of the new model-free optimisation protocol]{A schematic of the new model-free optimisation protocol.  Initially models $tm0$ to $tm9$ (\ref{model: tm0}--\ref{model: tm9}) of the set $\Localset_i$ for each spin system $i$ are optimised, model elimination used to remove failed models, and AIC model selection used to pick the best model.  Once all the $\Localset_i$ have been determined for the system the the local $\tau_m$ parameter is removed, the model-free parameters are held fixed, and the global diffusion parameters of $\Diffset$ are optimised.  These parameters are used as input for the central part of the schematic which follows the same procedure as that of Figure~\ref{fig: init diff estimate}.  Convergence is however precisely defined as identical models $\Space$, identical $\chi^2$ values, and identical parameters $\theta$ between two iterations.  The universal solution $\widehat\Univset$, the best description of the dynamics of the molecule, is determined using AIC model selection to select between the local $\tau_m$ models for all spins, the sphere, oblate spheroid, prolate spheroid, ellipsoid, and possibly hybrid models whereby multiple diffusion tensors have been applied to different parts of the molecule.} \label{fig: new protocol}
\end{figure}
