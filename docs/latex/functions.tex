% Program functions chapter.
%%%%%%%%%%%%%%%%%%%%%%%%%%%%

\chapter{Alphabetical listing of user functions}

The following is a listing with descriptions of all the user functions availible within the relax prompt and scripting environments.  These are simply an alphabetical list of the docstrings which can normally be viewed in prompt mode by typing `\texttt{help(function)}'.



% The help system.
%~~~~~~~~~~~~~~~~~

\section{The help system}

For assistance in using a function, simply type `\texttt{help(function)}'.  All functions can be viewed by hitting the [TAB] key.  In addition to functions, if `\texttt{help(object)}' is typed, the help for the python object is returned.  This system is similar to the help function built into the python interpreter, which has been renamed to \texttt{help\_python}, with the interactive component removed.  For the interactive python help system, type `\texttt{help\_python()}'.



% A warning about the formatting.
%~~~~~~~~~~~~~~~~~~~~~~~~~~~~~~~~

\section{A warning about the formatting}

The following documentation of the user functions has been automatically generated by a script which extracts and formats the docstring associated with each function.  There may therefore be instances where the formatting has failed or where there are inconsistencies.



% The list of functions.
%~~~~~~~~~~~~~~~~~~~~~~~

\section{The list of functions}

Each user function is presented within it's own subsection with the documentation broken into three parts, the synopsis, the default arguments, and the function's docstring.


% The synopsis.
\subsection{The synopsis}

The synopsis presents a brief description of the function.  It is taken as the first line of the docstring when browsing the help system.


% The default arguments.
\subsection{The default argments}

The default arguments list all the argments taken by the function.  To invoke the function, type the function name (tab completion is implemented to prevent insanity as the function names can be quite long -- a deliberate feature to improve usability), then in brakets type a comma separated list of arguments.

The first argument printed is always `self' but you can safely ignore it.  `self' is part of the object oriented programming within Python and is automatically prefixed to the list of arguments you supply.  Therefore you can't provide `self' as the first argument, even if you do try.

Two types of arguments exist in Python, standard arguments and keyword arguments.  The majority of arguments within the relax user functions are keyword arguments however you may, in rare cases, encounter a non-keyword argument.  For these standard arguments, just type the values in, although they must be in the correct order.  Keyword arguments consist of two parts, the key and the value.  For example the key may be \texttt{file} while the value you would like to supply is \texttt{`R1.out'}.  Various methods exist for supplying this argument.  Firstly you could simply type \texttt{`R1.out'} into the correct position in the argument list.  Secondly you can type \texttt{file=`R1.out'}.  The power of this second option is that argument order is unimportant.  Therefore if you would like to change the default value of the very last argument, you don't have to supply values for all other arguments.  The only catch is that standard arguments must come before the keyword arguments.


