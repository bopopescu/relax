%%%%%%%%%%%%%%%%%%%%%%%%%%%%%%%%%%%%%%%%%%%%%%%%%%%%%%%%%%%%%%%%%%%%%%%%%%%%%%%
%                                                                             %
% Copyright (C) 2006,2014,2017,2019 Edward d'Auvergne                         %
%                                                                             %
% This file is part of the program relax (http://www.nmr-relax.com).          %
%                                                                             %
% This program is free software: you can redistribute it and/or modify        %
% it under the terms of the GNU General Public License as published by        %
% the Free Software Foundation, either version 3 of the License, or           %
% (at your option) any later version.                                         %
%                                                                             %
% This program is distributed in the hope that it will be useful,             %
% but WITHOUT ANY WARRANTY; without even the implied warranty of              %
% MERCHANTABILITY or FITNESS FOR A PARTICULAR PURPOSE.  See the               %
% GNU General Public License for more details.                                %
%                                                                             %
% You should have received a copy of the GNU General Public License           %
% along with this program.  If not, see <http://www.gnu.org/licenses/>.       %
%                                                                             %
%%%%%%%%%%%%%%%%%%%%%%%%%%%%%%%%%%%%%%%%%%%%%%%%%%%%%%%%%%%%%%%%%%%%%%%%%%%%%%%


% Free software infrastructure chapter.
%%%%%%%%%%%%%%%%%%%%%%%%%%%%%%%%%%%%%

\chapter{Free software infrastructure} \label{ch: free software}

\section{History}

Starting with the initial code in November 2001, the relax sources were not stored within a version control repository.
Instead version control was performed by creating a log-less \file{*.tar.gz} file backup after each change.
In June 2005, these backup files were imported into a new Subversion (SVN) repository.

Corresponding with the switch to the SVN version control repository, relax was made public by shifting development onto the \href{https://en.wikipedia.org/wiki/Gna!}{Gna!\ free software infrastructure}.
This allowed for the following new infrastructure to be set up:
\begin{itemize}
    \item Hosting for the relax website.
    \item File download services for the relax distribution files (and PDF manual).
    \item The relax mailing lists.
    \item Access to the relax source code.
    \item Bug, support request, task, and patch trackers.
    \item News feed.
\end{itemize}
An archive of \href{https://web.archive.org/web/20170301004608/https://gna.org/projects/relax/}{relax's old Gna!\ website can be found on the Internet Archive}.

In May 2017, without much warning, the \href{https://en.wikipedia.org/wiki/Gna!}{Gna! infrastructure} that relax relied upon was permanently shut down.
This lead to a long period with no open source infrastructure.
During this time, the relax SVN repository was painstakingly converted into a git version control repository with all branches and history preserved.
To ensure that relax would be accessible for a long time into the future, this new git repository was mirrored to a number of free software/open source infrastructures:
\begin{itemize}
    \item \href{https://bitbucket.org/nmr-relax/}{relax at Bitbucket}
    \item \href{https://github.com/nmr-relax}{relax at GitHub}
    \item \href{https://gitlab.com/nmr-relax}{relax at GitLab}
    \item \href{https://sourceforge.net/projects/nmr-relax/}{relax at SourceForge}
\end{itemize}
The webpages were also migrated with history from the SVN repository where they were located alongside the relax source code to a separate git repository.

From January 2019, relax moved to the \href{https://sourceforge.net/projects/nmr-relax/}{SourceForge open source infrastructure}.  This provides the following relax infrastructure.
\begin{itemize}
    \item Hosting for the relax website.
    \item File download services for the relax distribution files (and PDF manual).
    \item The relax mailing lists.
    \item Access to the relax source code, web pages, and relax demo files.
    \item Bug, support request, and task trackers.
    \item SVN support for hosting the old and archived SVN repository.
    \item Backend shell log in (shell services).
    \item MySQL and PHP support (possibly allowing for the relax Mediawiki to be migrated here in the future).
\end{itemize}



% The relax web sites.
%~~~~~~~~~~~~~~~~~~~~~

\section{The relax web sites}
\index{web site|textbf}

The main web site for relax is \href{http://www.nmr-relax.com}{http://www.nmr-relax.com}.
From these pages general information about the program, links to the latest documentation, links to the most current software releases, and information about the mailing lists\index{mailing list} are available.
There are also Google\index{Google} search capabilities built into the pages for searching both the HTML version\index{manual!HTML} of the manual and the archives of the mailing lists\index{mailing list!archive}.



% The mailing lists.
%~~~~~~~~~~~~~~~~~~~

\section{The mailing lists}\label{sect: relax mailing lists}
\index{mailing list|textbf}

A number of mailing lists have been created covering different aspects of relax.
These include the announcement list\index{mailing list!relax-announce}, the relax users list\index{mailing list!relax-users}, the relax development list\index{mailing list!relax-devel}, and the relax committers list\index{mailing list!relax-commits}.


% relax-announce mailing list.
\subsection{relax-announce}

The relax announcement list ``\relaxAnnounceML''\index{mailing list!relax-announce} is reserved for important announcements about the program including the release of new program versions.
The amount of traffic on this list is relatively low.
The mailing list links are: \href{\relaxAnnounceMLSubscribeURL}{Subscribe}, \href{\relaxAnnounceMLArchiveURL}{Archive}\index{mailing list!archive}, \href{\relaxAnnounceMLSearchURL}{Search}.


% relax-users mailing list.
\subsection{relax-users} \label{sect: relax-users mailing list}

If you would like to ask questions about relax, discuss certain features, receive help, or to communicate on any other subject related to relax the mailing list ``\relaxUsersML''\index{mailing list!relax-users} is the place to post your message.
The mailing list links are: \href{\relaxUsersMLSubscribeURL}{Subscribe}, \href{\relaxUsersMLArchiveURL}{Archive}\index{mailing list!archive}, \href{\relaxUsersMLSearchURL}{Search}.


% relax-devel mailing list.
\subsection{relax-devel} \label{sect: relax-devel mailing list}

A second mailing list exists for posts relating to the development of relax.
Feature requests, program design, or any other posts relating to relax's structure or code should be sent to this list instead.
The list is ``\relaxDevelML''\index{mailing list!relax-devel} and the relevant links are:  \href{\relaxDevelMLSubscribeURL}{Subscribe}, \href{\relaxDevelMLArchiveURL}{Archive}\index{mailing list!archive}, \href{\relaxDevelMLSearchURL}{Search}.


% relax-commits mailing list.
\subsection{relax-commits}

One last mailing list is the relax commits list\index{mailing list!relax-commits}.
This list is reserved for automatically generated posts created by the version control software which looks after the relax source code and these web pages.
If you would like to become a developer, please subscribe to this list.
The mailing list links are:  \href{\relaxCommitsMLSubscribeURL}{Subscribe}, \href{\relaxCommitsMLArchiveURL}{Archive}\index{mailing list!archive}, \href{\relaxCommitsMLSearchURL}{Search}.


% Replying to a message.
\subsection{Replying to a message}

When replying to a message on these lists remember to hit `respond to all' so that the mailing list is included in the CC field.
Otherwise your message will only be sent to the original poster and not return back to the list.
Only messages to relax-users and relax-devel will be accepted.
If you are using Gmail's web based interface, please do not click on `Edit Subject' as this currently mangles the email headers, creates a new thread on the mailing list, and makes it difficult to follow the thread.



% Reporting bugs.
%~~~~~~~~~~~~~~~~

\section{Reporting bugs}\label{reporting bugs}
\index{bug|textbf}

One of the philosophies in the construction of relax is that if there is something which is not immediately obvious then that is considered a design bug\index{bug!design}.
If any flaws in relax are uncovered including general design flaws, bugs in the code, or documentation issues these can be reported within \href{\relaxBugTrackerURL}{relax's bug tracker system}\index{bug!tracker}.
Please \href{\relaxBugTrackerSubmitURL}{submit a relax bug here} rather than reporting bugs to personal email addresses or to the mailing lists.

When reporting a bug please include as much information as possible so that the problem can be reproduced.
Include information such as the release version or the revision number if the repository sources are being used.
Also include all the steps performed in order to trigger the bug.
Attachment of files is allowed so scripts and subsets of the input data can be included.
However please do not attach large files to the report.
Prior to reporting the bug try to make sure that the problem is indeed a bug and if you have any doubts please feel free to ask on the relax-users mailing list\index{mailing list!relax-users}.
To avoid duplicates be sure that the bug has not already been submitted to the bug tracker\index{bug!tracker}.
You can \href{\relaxBugTrackerSearchURL}{search through the bugs here}\index{bug!search}.

Once the bug has been confirmed by one of the relax developers you may speed up the resolution of the problem by trying to fixing the bug yourself.
If you do wish to play with the source code and try to fix the issue see the relax development chapter of this manual on how to check out the latest sources (Chapter~\ref{ch: relax devel} on page~\pageref{ch: relax devel}), how to generate a patch (which is just the output of diff\index{diff} in the `unified' format), and the guidelines for the format of the code.



% Latest sources -- the relax repositories.
%~~~~~~~~~~~~~~~~~~~~~~~~~~~~~~~~~~~~~~~~~~

\section{Latest sources -- the relax repositories}
\index{repository|textbf}

relax's source code is kept within a version control system called \href{https://git-scm.com/}{git}\index{git|textbf}.
This system allows for fine control over the development of the program.
The repository contains all information about every change ever made to the program.
To learn more about the system, the \href{https://git-scm.com/docs}{git Reference Manual}\index{git!manual} is a good place to start.
The \href{\relaxCodeWebInterfaceURL}{contents of the relax repository can be viewed online}.
The current sources can be downloaded using the git protocol by typing

\example{\$ git clone \relaxCodeURL\ relax}
\index{git!clone}



% The relax distribution archives.
%~~~~~~~~~~~~~~~~~~~~~~~~~~~~~~~~~

\section{The relax distribution archives}
\index{distribution archive}

The relax distribution archives are the files to \href{\relaxDistributionURL}{download to install relax}.
If a compiled binary distribution for your architecture does not exist, you are welcome to create this distribution yourself and submit it for inclusion in the relax project.
To do this a number of steps are required.
Firstly, the code to each relax release or version resides in a git repository `tag'.
To check out version 4.0.3, for example, within an existing git repository clone type

\example{\$ git checkout 4.0.3}
\index{git!checkout}

The binary distribution can then be created for your architecture by typing

\begin{exampleenv}
\$ scons binary\_dist
\end{exampleenv}
\index{SCons!binary distribution}

At the end SCons\index{SCons} will attempt to make a GPG signature\index{GPG!signature} for the newly created archive.
However this will fail as the current relax private GPG key\index{GPG!key} is not available for security reasons.
If the SCons command fails, excluding the GPG signing, please \href{\relaxBugTrackerSubmitURL}{submit a bug report}\index{bug!tracker} with as much information possible (the Python and SCons version numbers may also be useful).
Once the file has been created post a message to the relax development mailing list\index{mailing list!relax-devel} describing the compilation and the creation of the archive, the relax version number, the machine architecture, operating system, and the name of the new file.
Do not attach the file though.
You will then receive a response explaining where to send the file to.
For security the archive will be thoroughly checked and if the source code is identical to that in the repository and the C modules are okay, the file will be GPG signed\index{GPG!signature} and uploaded to \href{\relaxDistributionURL}{\relaxDistributionURL}.
