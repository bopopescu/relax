% Installation instructions.
%%%%%%%%%%%%%%%%%%%%%%%%%%%%

\chapter{Installation instructions}


% Dependencies.
%~~~~~~~~~~~~~~

\section{Dependencies}

The following packages need to be installed before using relax:

\begin{description}
\item[\href{http://python.org/}{Python}\index{Python}:]  Version 2.4 or higher (although any 2.x version may work).
\item[\href{http://numpy.scipy.org/}{Numeric}\index{Numeric}:]  Version 21 or higher.
\item[\href{http://starship.python.net/~hinsen/ScientificPython/}{ScientificPython}\index{ScientificPython}:]  Version 2.2 or higher.
\item[Optik\index{Optik}:]  Version 1.4 or higher.  This is only needed if running python $<=$ 2.2.
\end{description}

Older versions of these packages may work, use them at your own risk.  If, for older dependency versions, errors do occur please submit a bug report to the bug tracker\index{bug tracker} at \href{https://gna.org/bugs/?group=relax}{https://gna.org/bugs/?group=relax}.  That way a solution may be created for the next relax release.



% Installation.
%~~~~~~~~~~~~~~

\section{Installation}
\index{installation|textbf}


% The precompiled verses source distribution.
\subsection{The precompiled verses source distribution}

Two types of software packages are available for download -- the precompiled and source distribution.  Currently only relaxation curve-fitting requires compilation to function and all other features of relax will be fully functional without compilation.  If relaxation curve-fitting is required but no precompiled version of relax exists for your operating system or architecture then, if a C compiler is present, the C code can be compiled into the shared objects files \texttt{*.so} which are loaded as modules into relax\index{C module compilation}.  To build these modules the Sconstruct\index{Sconstruct|textbf} system from \href{http://scons.org/}{http://scons.org/} is required.  This software only depends on Python which is essential for running relax anyway.  Once Sconstruct is installed type

\example{\$ scons}
\index{scons}

in the base directory where relax has been installed and the C modules should, hopefully, compile without any problems.  Otherwise please submit a bug report to the bug tracker\index{bug tracker} at \href{https://gna.org/bugs/?group=relax}{https://gna.org/bugs/?group=relax}.



% Installation on GNU/Linux.
\subsection{Installation on GNU/Linux}
\index{GNU/Linux|textbf}

To install the program relax on a GNU/Linux system download either the precompiled distribution\index{distribution archive} labelled \texttt{relax-x.x.x.GNU-Linux.\textit{arch}.tar.bz2} matching your machine architecture or the source distribution \texttt{relax-x.x.x.src.tar.bz2}.  A number of installation methods are possible.  The simplest way is to switch to the user `root', unpack and decompress the archive within the \texttt{/usr/local} directory by typing, for instance

\example{\$ tar jxvf relax-x.x.x.GNU-Linux.i686.tar.bz2}
\index{tar}

then create a symbolic link in \texttt{/usr/local/bin} by moving to that directory and typing

\example{\$ ln -s ../relax/relax .}
\index{symbolic link}

and finally running relax to create the byte-compiled Python \texttt{*.pyc} files to speed up the start time of relax by typing

\example{\$ relax --test}

Alternatively if the Sconstruct system is installed typing

\example{\$ scons install}
\index{scons}

in the relax base directory will create a directory in \texttt{/usr/local/} called \texttt{relax}, copy all the uncompressed and untarred files into this directory, create a symbolic link in \texttt{/usr/local/bin} to the file \texttt{/usr/local/relax/relax}, and then finally run relax to create the byte-compiled Python \texttt{*.pyc}.  To change the installation path to a non-standard location the Sconstruct script \texttt{sconstruct} in the base relax directory should be modified by changing the variable \texttt{INSTALL\_PATH} to point to the desired location.



% Installation on MS Windows.
\subsection{Installation on MS Windows}
\index{MS Windows|textbf}

In addition to the above dependencies, running relax on MS Windows requires a number of additional programs.  These include:

\begin{description}
\item[\href{http://projects.scipy.org/ipython/ipython/wiki/PyReadline/Intro}{pyreadline}\index{pyreadline}:]  Version 1.3 or higher (\href{http://ipython.scipy.org/dist/pyreadline-1.3.win32.exe}{download}).
\item[\href{http://starship.python.net/crew/theller/ctypes/}{ctypes}\index{ctypes}:]  The pyreadline package requires ctypes (\href{http://prdownloads.sourceforge.net/ctypes/ctypes-1.0.0.win32-py2.4.exe?download}{download}).
\end{description}

To install, simply download the pre-compiled binary distribution \texttt{relax-x.x.x.Win32.zip} or the source distribution \texttt{relax-x.x.x.src.zip} and extract the files to \texttt{C:$\backslash$Program Files$\backslash$relax-x.x.x}.  Then add this directory to the system environment path (in Windows XP, right click on `My Computer', go to `Properties', click on the `Advanced' tab, and click on the `Envirnment Variables' button.  Then double click on the `Path' system variable and add the text ``;C:$\backslash$Program Files$\backslash$relax-x.x.x'' to the end of variable value field.  The Python installation must also be located on the path (add the text ``;C:$\backslash$Program Files$\backslash$Python24'', changing the text to point to the correct directory, to the field).  To run the program from any directory inside the Windows command prompt (or dos prompt) type:

\example{C:$\backslash$> relax}




% Installation on Mac OS X.
\subsection{Installation on Mac OS X}
\index{Mac OS X|textbf}

Please write me if you know how to do this!



% Installation on your OS.
\subsection{Installation on your OS}

Please write me if you know how to do this!



% Running a non-compiled version.
\subsection{Running a non-compiled version}

Compilation of the C code is not essential for running relax, however certain features of the program will be disabled.  Currently only the exponential curve-fitting for determining the $\Rone$ and $\Rtwo$ relaxation rates requires compilation.  To run relax without compilation install the dependencies detailed above, download the source distribution which should be named \texttt{relax-x.x.x.src.tar.bz2}, extract the files, and then run the file called \texttt{relax} in the base directory.



% Optional programs.
%~~~~~~~~~~~~~~~~~~~

\section{Optional programs}

The following is a list of programs which can be used by relax although they are not essential for normal use.


% Grace.
\subsection{Grace}
\index{computer programs!Grace|textbf}

Grace is a program for plotting two dimensional data sets in a professional looking manner.  It is used to visualise parameter values.  It can be downloaded from \href{http://plasma-gate.weizmann.ac.il/Grace/}{http://plasma-gate.weizmann.ac.il/Grace/}.


% OpenDX.
\subsection{OpenDX}
\index{computer programs!OpenDX|textbf}

Version 4.1.3 or compatible.  OpenDX is used for viewing the output of the space mapping function and is executed by passing the command \texttt{dx} to the command line with various options.  The program is designed for visualising multidimensional data and can be found at \href{http://www.opendx.org/}{http://www.opendx.org/}.


% Molmol.
\subsection{Molmol}
\index{computer programs!Molmol|textbf}

Molmol is used for viewing the PDB structures loaded into the program and to display parameter values mapped onto the structure.


% PyMOL.
\subsection{PyMOL}
\index{computer programs!PyMOL|textbf}

PDB structures can also be viewed using PyMOL.  Although the mapping of parameter values onto the structure is not yet supported, this program can be used to display geometric objects generated by relax for representing physical concepts such as the diffusion tensor.


% Dasha.
\subsection{Dasha}
\index{computer programs!Dasha|textbf}

Dasha is a program used for model-free analysis of NMR relaxation data.  It can be used as an optimisation engine to replace the minimisation algorithms implemented within relax.


% Modelfree4.
\subsection{Modelfree4}
\index{computer programs!Modelfree4|textbf}

Art Palmer's Modelfree4 program is also designed for model-free analysis and can be used as an optimisation engine to replace relax's high precision minimisation algorithms.
